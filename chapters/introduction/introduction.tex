\chapter{Introduction}

\section{Purpose of the Study}

The main purpose of this thesis is to incorporate the two research areas \mss{}
and \ogs{}. Both areas have individually been the topic of intense research
lately \todo{Citation needed} but no research exists on how to combine the two.

In regard to the \mss{} domain this thesis serves as a base for discussion about
\ms{} applications. Specifically the development and operation a set of \mss{} is
explained in detail. Of particular interest is the topic of how to facilitate
the composition of \mss{} and how to deploy a \ms{} application in a modern could
environment.

This thesis also provides a proof of concept that it is possible to
develop \ogs{} using \mss{}. For this purpose a vertical slice prototype is
developed to showcase the necessary fragments that make up a working \og{} in
the \ms{} world. Furthermore the prototype serves as a simple development
example of how to reproduce the development process of an \og{}. Is intended to
live on after this thesis ends and will continuously improved in the future.

As a secondary objective the usability of the recently very popular
\todo{Citation needed} Procedural Content Generation (PCG) is also discussed in
this thesis.

\section{Theoretical framework}

To curtail the infinite amount of research problems that the topics \ogs{} and
\mss{} offer a theoretical framework can be used. Within the framework the
abstract research problems are translated to research questions that cover the
area of interest sufficiently. According to the research questions hypotheses
can be formulated. The verification of these hypotheses is the final goal of
this thesis.

\subsection{Research Problems and Research Questions}

\subsubsection{Usability of a \ms{} for \ogs{}} 

\paragraph{Research Problem:} The usability of \ms{} suited to develop and
operate an \og{}.

\paragraph{Research Questions:}
\begin{itemize}
  \item Can an \og{} run in a pure \ms{} environment?
  \item Is a \ms{} influenced architecture suitable to design an \og{}?
  \item Is the added complexity during development tolerable?
  \item Is the overhead introduced by \mss{} negligible? 
\end{itemize}

\subsubsection{Deployment and Operation}

\paragraph{Research Problem:} The deployment and operation of a \ms{}
application in a cloud cluster-environment.

\paragraph{Research Questions:}
\begin{itemize}
  \item Which steps are necessary to operate an \og{} in a cluster-environment?
  \item How well do \ogs{} operate in a cluster-environment?
  \item How does a modern container engine assist in building and operating a
  \ms{} driven application?
  \item How can the deployment process be automated?
  \item Which tools exist that help the developer with deployment?
\end{itemize}

\subsubsection{\ms{} Composition}
\paragraph{Research Problem:} The composition of a set of \mss{} to form a coherent
distributed application.

\paragraph{Research Questions:}
\begin{itemize}
  \item What degree of composition is necessary for a \ms{} application to work?
  \item What is the minimal set of static and dynamic information that is needed
  for \ms{} composition?
  \item Which of the common paradigms orchestration and choreography is suited
  better for \ms{} composition?
  \item Which middle-ware is most useful in providing \ms{} composition?
\end{itemize}

\subsubsection{\ms{} Coupling}
\paragraph{Research Problem:} The degree to which \mss{} are semantically
dependent on each other. 

\paragraph{Research Questions:}
\begin{itemize}
  \item How can multiple game features be developed in parallel in different
  \mss{} while preserving the overall application behaviour?
  \item What problems are introduced by coupling game features loosely? 
  \item How does the consistency vs. performance trade-off impact \ogs{}?
  \item What protocols help to reduce service coupling?
  \item Can global static domain information help to reduce \ms{} coupling?
\end{itemize}

\subsubsection{\ms{} Development}
\paragraph{Research Problem:} The development process of a \ms{} application.

\paragraph{Research Questions:}

Development of a \ms{} application:
\begin{itemize}
  \item What separates \ms{} development from regular development?
  \item Which is the minimal number of steps ``from code to cloud''?
  \item Which tools help with \ms{} development?
  \item Which development tools make the development process significantly
  simpler?
\end{itemize}

\subsection{Hypotheses}

The essence of the research problems can be condensed to a set of hypotheses
which correspond with the different research areas. All hypotheses are
verified in the course of this thesis to prove that \mss{} are suited to
build an \og{}. 

\subsubsection{\ogs{} and \mss{} Hypothesis}
It is possible to build an \og{} as a pure \ms{} application by
respecting all seven \ms{} tenets.

\subsubsection{Required Features Hypothesis}
A minimal set of features is required to build a distributed \og{}: Networking,
persistence, serialization, session management and game engine integration.

\subsubsection{Stability and Performance Hypothesis} 
Microservices allow to separate the performance critical real-time game
simulation in regards to computing, bandwidth and latency requirements from the
non time-critical backend functionality of an online game. This allows arbitrary
scaling.

\subsubsection{Simple Development Hypothesis} 
Microservice can make the complicated process of developing online games simpler
by splitting up a complex game domain into small parts that can be developed
independently.

\subsubsection{Reproducibility Hypothesis} 
It is possible to completely prevent initial costs to kick of the development
process using only free/open-source technologies. This allows to reproduce the
development process anywhere at any time.

\section{Significance of the Study}

\mss{} are a very modern approach to realize a Service Oriented Architecture
(SOA). While SOA is not something new and has been in use for more then a decade
\todo{Citation needed} \mss{} are a cutting edge implementation approach to SOA
which has gained popularity only in the recent years. Therefore the theory and
references of how to actually develop and operate such a \ms{} environment are
rather sparse. This thesis servers as a complete reference for all relevant
aspects that have to be taken into account when an application (especially an
\og{}) is developed using \mss{}.

In regards to \ogs{} \mss{} have gained recent popularity. This is due to the
fact that large game development companies (>100 developers) prosecute \ogs{}
with million of player playing at the same time. For these games to scale the
traditional approach of running an \og{} as a monolithic application is not
sufficient. These large companies have large budgets at their disposal to
realize such an \og{}.

But this is not the case for smaller independent game development companies.
Independent companies are not constraint by founders and are therefor the main
driver for innovation in the game industry at the moment. But these  indy-teams don't
have the manpower to develop rich \ogs{} from scratch because the lack of
manpower and financial resources. Also AAA companies don't share their knowledge
about \og{} development to an extent that would allow indies to develop rich
\ogs{}.

Further \og{} development introduces the aspect of distribution within the game
application which is by definition a hard problem \todo{Citation needed}. There
is a lot of existing information on how to cope with asynchronism in game
development\todo{Cite: Gambetta}. However these references are mostly very low
level and leave a lot of responsibility to the developer. This added complexity
is whats holding back creative minds to develop rich \ogs{}. This thesis should
serve as a reference for these independent developers to gain a foothold in the
complex area of \og{} development.

\section{Definitions}

\section{Terminology Definitions}

\subsection{Simple Online Game}

A Online Game where where players are responsible to set up game servers. In
these environments the players make the game rules. Usually those are small scale
games which are played in a couple of session and the game state is saved
locally on the client. Examples are old LAN style games like for example Quake,
Age of Empires, or Diablo 2. If any online matchmaking exists for these games,
there is no centralized rule enforcement system in place.

\subsection{Rich Online Game}

These online games only require the player to download the game client or eaven
to just play it in the browers. That Game state of all players is stored on the
server infrastructure. A part from the game experience comes from this aspect of
an online managed game state. The rules must be enforced on the server
environment.

\section{Matchmaking}

Matchmaking is a process that helps players to find each other for an online
game. 


\section{Delimitations, Limitations, and Assumptions}