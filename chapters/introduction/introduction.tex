\chapter{Introduction}

\section{Purpose of the Study}

The main purpose of this thesis is to incorporate the two research areas \mss{}
and \ogs{}. Both areas have individually been the topic of intense research
lately \todo{Citation needed} but no research exists on how to combine the two.

In regard to the \mss{} domain this thesis serves as a base for discussion about
\ms{} applications. Specifically the development and operation a set of \mss{} is
explained in detail. Of particular interest is the topic of how to facilitate
the composition of \mss{} and how to deploy a \ms{} application in a modern could
environment.

This thesis also provides a proof of concept that it is possible to
develop \ogs{} using \mss{}. For this purpose a vertical slice prototype is
developed to showcase the necessary fragments that make up a working \og{} in
the \ms{} world. Furthermore the prototype serves as a simple development
example of how to reproduce the development process of an \og{}. Is intended to
live on after this thesis ends and will continuously improved in the future.

As a secondary objective the usability of the recently very popular
\todo{Citation needed} Procedural Content Generation (PCG) is also discussed in
this thesis.

\section{Theoretical framework}

To curtail the infinite amount of research problems that the topics \ogs{} and
\mss{} offer a theoretical framework can be used. Within the framework the
abstract research problems are translated to research questions that cover the
area of interest sufficiently. According to the research questions hypotheses
can be formulated. The verification of these hypotheses is the final goal of
this thesis.

\subsection{Research Problems and Research Questions}

\subsubsection{Usability of a \ms{} for \ogs{}} 

\paragraph{Research Problem:} The usability of \ms{} suited to develop and
operate an \og{}.

\paragraph{Research Questions:}
\begin{itemize}
  \item Can an \og{} run in a pure \ms{} environment?
  \item Is a \ms{} influenced architecture suitable to design an \og{}?
  \item Is the added complexity during development tolerable?
  \item Is the overhead introduced by \mss{} negligible? 
\end{itemize}

\subsubsection{Deployment and Operation}

\paragraph{Research Problem:} The deployment and operation of a \ms{}
application in a cloud cluster-environment.

\paragraph{Research Questions:}
\begin{itemize}
  \item Which steps are necessary to operate an \og{} in a cluster-environment?
  \item How well do \ogs{} operate in a cluster-environment?
  \item How does a modern container engine assist in building and operating a
  \ms{} driven application?
  \item How can the deployment process be automated?
  \item Which tools exist that help the developer with deployment?
\end{itemize}

\subsubsection{\ms{} Composition}
\paragraph{Research Problem:} The composition of a set of \mss{} to form a coherent
distributed application.

\paragraph{Research Questions:}
\begin{itemize}
  \item What degree of composition is necessary for a \ms{} application to work?
  \item What is the minimal set of static and dynamic information that is needed
  for \ms{} composition?
  \item Which of the common paradigms orchestration and choreography is suited
  better for \ms{} composition?
  \item Which middle-ware is most useful in providing \ms{} composition?
\end{itemize}

\subsubsection{\ms{} Coupling}
\paragraph{Research Problem:} The degree to which \mss{} are semantically
dependent on each other.

\paragraph{Research Questions:}
\begin{itemize}
  \item How can multiple game features be developed in parallel in different
  \mss{} while preserving the overall application behaviour?
  \item What problems are introduced by coupling game features loosely? 
  \item How does the consistency vs. performance trade-off impact \ogs{}?
  \item What protocols help to reduce service coupling?
  \item Can global static domain information help to reduce \ms{} coupling?
\end{itemize}



Development of a \ms{} application:
\begin{itemize}
  \item Usability 
\end{itemize}


\subsection{Hypotheses}

To be more specific about the validation it must be defined what knowledge is
elaborated in this thesis. This is described by asking the major knowledge
questions.






\section{Significance of the Study}

Today we we see a large number of game development companies. There is a large
number of companies that have established themselves over the last year
\todo{Citation Needed.}. These triple A companies (AAA) have the budget to
assign hundreds of workers to game projects with a budged of millions of
dollars \todo{Cite to GTA5}.\\

In the contrary we have many small independent that work in small teams. The
keep the right on the intellectual property of their games and are not
restricted by any financial backers. These companies need other types of
financing.\todo{Backup with facts}. These factors limit the companies in the
extent they can do games in.\\

Hence indy-teams don't have the manpower to develop rich online game experiences
from scratch because they lack the manpower. The fact that AAA companies don't
share their knowledge about online game development further complicates this
problem. The lack of tools in this area is another reason that indies struggle
with dense online games. On the needed level of detail simply no middle ware
exists.

\section{Definitions}

\section{Terminology Definitions}

\subsection{Simple Online Game}

A Online Game where where players are responsible to set up game servers. In
these environments the players make the game rules. Usually those are small scale
games which are played in a couple of session and the game state is saved
locally on the client. Examples are old LAN style games like for example Quake,
Age of Empires, or Diablo 2. If any online matchmaking exists for these games,
there is no centralized rule enforcement system in place.

\subsection{Rich Online Game}

These online games only require the player to download the game client or eaven
to just play it in the browers. That Game state of all players is stored on the
server infrastructure. A part from the game experience comes from this aspect of
an online managed game state. The rules must be enforced on the server
environment.

\section{Matchmaking}

Matchmaking is a process that helps players to find each other for an online
game. 

\section{Problem Research}

It would be best if every research would be done completely platform
independent. This would inquire very general approaches to document and state
the findings. \todo{cite to Lewis} Lewis said that if you have implemented
something then you have understood it. Since modern programming languages are
turing complete it is proven that it is possible to apply the developed concepts 
also to other programming languages.
\question{Ist dies legitim? Es w�rde mir enorm helfen.}

\section{Delimitations, Limitations, and Assumptions}