\section{Procedures}

The research methods used to examine the research problems explained in
\autoref{sec:theoretical_framework} can be categorized into four areas:
Literature research, lab research, field research, and statistical analysis.

Each category provides a framework suitable to enable research for one or more
of the concrete research subjects defined in \autoref{sec:subject_pool}. The
\autoref{tab:research_methods_problems} below shows which research methods are
used for each research subject.

\begin{table}

	\begin{center}
	  \begin{tabular}{ l | c | c | c | c | }
	  
	  	&\begin{turn}{90}\textbf{Literature Research}\end{turn}
	  	&\begin{turn}{90}\textbf{Lab Research}\end{turn}
	  	&\begin{turn}{90}\textbf{Field Research}\end{turn}
	  	&\begin{turn}{90}\textbf{Statistical Research}\end{turn}
	  	\\\hline
	    
	    \textbf{Challenges Introduced by Microservices}&&x&x&\\\hline
	    \textbf{Microservice Infrastructure}&x&x&&\\\hline
	    \textbf{Networking in Online Games}&&&x&x\\\hline
	    \textbf{Development of a Microservice Driven Online Game}&&x&x&x\\\hline
	    \textbf{Procedural Content Generation in Online Games}&x&&&\\\hline
	  \end{tabular}
	\end{center}
	\caption{Research Methods used to tackle research problems.}
	\label{tab:research_methods_problems}
\end{table}
The specific methods used in each category are derived from the design
science methodology principles \cite{wieringa2014design_science}.

\subsection{Literature Research}

Literature only plays a minor role in this thesis because the theoretical
foundation on \mss{} and \ogs{} has already been established in the last two
project theses. Nonetheless literature is consulted in relevant areas to give
a theoretical backing to findings of this thesis.

Especially in the topic of \ms{} infrastructure a deep study of the
documentation of composition engines and cloud service providers was made
to highlight advantages and differences.

Since much sophisticated information about the \mss{} is shared via articles
and blog posts this type of references is considered valuable to back findings
with opinions of domain experts in the software industry.

In regards to \gls{pcg} the book Procedural Content Generation In
Games \cite{shaker2014procedural} was studied in detail to identify \gls{pcg}
methods which are usable for \og{} development.

\subsection{Lab Research}
\label{sub:lab_reserach}

Lab research is a \gls{dsm} discipline to develop software by designing and
testing it under idealized laboratory conditions
\cite{wieringa2014design_science}. The majority of the research conducted during
this thesis is lab research. This is due to the very technical nature of the
subject.

The method used in lab research is the same for all research subjects namely
single-case experiments \cite{wieringa2014design_science}. Single-case
experiments examine one aspect of a problem domain under isolation.

Single-case experiments are conducted to examine all technical aspects of \mss{}
in regard to \ogs{} in a practical way. This includes the examination of the
\ms{} infrastructure requirements listed in \autoref{sub:infrastructure}, the
composition of \mss{}, the deployment of \mss{}, and the challenges introduced
by \mss{}.

All technical aspects can be examined on the basis of the instrumentation
artifact described in \autoref{sub:instrumentation}.
\autoref{tab:instrumentation_aspects} below shows which instrumentation artifact
is used to examine each technical aspect.

\begin{table}
	\begin{center}
	  \begin{tabular}{ l | l | l | l | l | l | }
	    &\begin{turn}{90}\textbf{Composition}\end{turn}
	    &\begin{turn}{90}\textbf{Deployment}\end{turn}
	    &\begin{turn}{90}\textbf{Persistence}\end{turn}
	    &\begin{turn}{90}\textbf{Consistency}\end{turn}
	    &\begin{turn}{90}\textbf{Monitoring}\end{turn}
	    \\\hline
	    
	    
	    \textbf{MicroNet}&x&x&x&x&x\\\hline
	    \textbf{Java as Programming Language}&x&&&&\\\hline
	    \textbf{Application Data Format}&x&&x&x&\\\hline
	    \textbf{Networking Technologies}&x&&&x&\\\hline
	    \textbf{Container Engine}&x&x&&&\\\hline
	    \textbf{Cloud Service Providers}&x&x&x&x&x\\\hline
	    \textbf{Composition Engines}&x&x&&&x\\\hline
	    \textbf{Database Solution}&&&x&x&\\\hline
	    \textbf{Source and Artifact Control}&&x&&&\\\hline
	    \textbf{Cloud Service Provider Gaming Solutions}&x&x&x&x&x\\\hline
	    \textbf{Dedicated Server}&x&x&x&x&x\\\hline
	    \textbf{Game Engine}&x&x&x&x&x\\\hline
	    \textbf{Spacegame Prototype}&x&x&x&x&x\\\hline
	  \end{tabular}
	\end{center}
	\caption{Instrumentation artifacts used to examine all technical aspects.}
	\label{tab:instrumentation_aspects}
\end{table}


\subsection{Field Research}

Field research is another \gls{dsm} technique and is the next step after
successful lab research to further back the findings gathered during the lab
research. Field research investigates how implemented artifacts interact with
their real-world context \cite{wieringa2014design_science}. In this thesis field
research was only possible to a limited extent because it is very preparation
intensive.

\subsubsection{Technical Action Research}

\gls{tar} is used to test a designed solution in a real world context
\cite{wieringa2014design_science}. This approach gives the researcher a deep
insight into the real world problem domain. A second result of \gls{tar} is that
the developed concepts or designs may be used to help a stakeholder in real
world projects.

\gls{tar} has been conducted in this thesis in the form of a series of
play-tests based on the MicroNet and the Spacegame Prototype.

\subsubsection{Expert Opinion}

The simplest way to validate a developed artifact is submitted to a panel of
experts with sufficient knowledge of the problem domain for evaluation. The
experts attempt to imagine how such an artifact will interact with the
real-world problem context \cite{wieringa2014design_science}. 

Expert opinion is used to validate the proposed development approach of
MicroNet. The experts solve a series of simple tasks with MicroNet and after
that share their impressions in a questionnaire. The chosen experts are master
students, game developers and software engineers.

\subsection{Statistical Analysis}

Since this thesis is mainly on an exploratory basis the gathering of empirical
data has proven to be difficult. Especially the gathering of operational data from
field research is very cumbersome and also requires a lot of effort to collect
the data and extract any meaningful information from it. Because of this
reason the statistical analysis will be limited to the evaluation of the data
gathered during the expert opinion validation.






