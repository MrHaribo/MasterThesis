\section{Procedures}

\section{Method Selection for Research Problems}

The subject pool discussed in \autoref{sec:subject_pool} are very variegated.
For the examination of the subjects it is therefor necessary to access each
subject with in a specifically suited method. The research problems can be
summarized into four problem domains which allows to choose the the optimal
research method for each domain individually.

\subsection{Composition}
Expert Opinion on Model Approach

\subsection{Deployment}
Evaluation of 3rd party libraries (DevOps work)


\subsection{Development}
Expert Opinion

A one click solution for developers to start with online game
  	  development. The development process should feel as closely as possible to
  	  offline game development.

\subsection{Operations}
Play Tests

\subsection{Lab Research}

\subsection{Eclipse}

Since MicroNet makes usage of various thirs party libraries and tools there
needs to be a way to manage them all. This purpose solves the Eclipse tools for
MicroNet. The MicroNet Tools is a set of plug-ins that are partially independent
of each other. They all help the developer to implement game specific \mss{}.

Since the UI of an Eclipse plug-in is written with the SWT library it can easily
be exported to a standalone tool that does not require Eclipse. However any code
related feature do require the Eclipse IDE.

\subsubsection{Service Generation}

Allows the developer to not worry about boiler plate code of setting up the
service in the framework. This covers the setup of the appropriate networking
solution according to the environment. This allows the developer to focus on the
actual service code.

\subsubsection{Code Assist}

The Code Assist plug-in helps the developer to keep track of functionality
provided by other services. It allows a type-safe communication between
services. The information that is needed to give these API proposals is
extracted from the service implementation at compile time. The information is
then shared via the version control system. This process is manually and the
developer must check in and probably merge his changes. 

The format of the distribute API description is not relevant because it is a
machine-readable format and the developer sees it in a polished form.

\subsubsection{Management UI}

The management UI plug-in helps the developer to get an overview over the whole
game application. It helps the manage the versions of services of the game
services as well as the versions of dependencies. This plug-in highly relies on
the description of the maven projects.

\subsection{Field Research}
