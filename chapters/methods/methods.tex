\chapter{Research Methods}

This thesis is the third and final part of a three semester project. The
substance of the past two theses was to build a foundation for the incorporation
of \mss{} and \ogs{}. 

The first project thesis \cite{biedermann2015project1} mostly consisted of a
literature research about the topics of \ogs{} and \mss{}. The result is a
comprehensive documentation which examines many aspects of both topics in
detail. The practical part of the first thesis was to develop a fully functional
\ms{} networking solution which suffices \og{} requirements.

In the second project thesis \cite{biedermann2016project2} the theoretical
foundation established in thesis one was examined in practice by developing
\og{} functionality on top of the networking foundation. The major focus during
thesis two was to examine what needs to happen inside \mss{} to realize \og{}
functionality while neglecting most external infrastructure aspects. The result
was a set of common \og{} features realized using \mss{}.

This third and final thesis examines how to build a coherent application with
a set of individual \mss{}. Especially composition and deployment
concepts are introduced as the necessary glue to bring a distributed application
together. In regards to composition, the orchestration and choreography concepts
are examined in detail and deployment is investigated by examining well known
containerization technologies.

An additional goal of thesis three is the elaboration of the previous two theses
to build a complete picture about \ms{} driven \ogs{}. This includes finalizing
the existing prototype and extend it with necessary features for composition and
deployment of \ms{} applications. The concepts developed in thesis three are
also validated through a user survey.

To give this thesis a scientific foundation \gls{dsm} principles are applied
whenever appropriate.

\section{Subject Pool}
\label{sec:subject_pool}

The research problems listed in \autoref{sub:problems} which describe the
subjects of this thesis are of great variation. In order to gain an overview
over all the activities that have to be conducted for the examination of the
research problems concrete research subjects can be formed. These subjects are
specifically defined to be well suited for individual examination.

\subsection{Challenges Introduced by \mss{}}
\label{sub:ms_challenges}

One main aspect to verify the \ogs{} and \mss{} hypothesis
(\autoref{sub:hypothesis}) is to find \ms{} friendly solutions for the major
challenges which \og{} development introduces. A solution can be considered
being \ms{} friendly if it suffices all seven \ms{} tenets.

\subsubsection{State Isolation}

It is most noticeable that \ms{} state isolation represents a major challenge
according to the IDEAL tenet. This is because a game is always
state-full\footnote{A game always has a state, like for example the positions of
all chess pieces on a chess board is the state of the chess game.}. Since \mss{}
are always state-less a solution must be found on how to make the game state
available to \mss{}.

This challenge is common in computer science disciplines and is referred to as
session management. In the case of \ogs{} sessions of players have to be
tracked, but also the sessions of the game itself (e.g. one game round).

\subsubsection{Model Access}

The state isolation challenge inevitably leads to another challenge namely the
game model access. Since the game state is shared among multiple \mss{} a
flexible approach is needed to provide access to session data the services
require.

\subsection{\ms{} Infrastructure}
\label{sub:infrastructure}

The \ms{} infrastructure subject is a collection of all topics that have to be
considered when developing and operating a \ms{} environment. The content of
this subject is not restricted to the \ogs{} and can be applied to any domain. 

\subsubsection{Requirements for \ms{} Composition}

The examination of the requirements for \ms{} composition can be grouped into
logical and physical composition requirements as explained in
\autoref{subsub:composition}.

The requirements of physical composition demand that \mss{} need a way to find
each other regardless where they are deployed. This requirement is tightly
coupled with physical characteristics of the target environment. The
requirements for physical composition of \ms{} \og{} environments are very
similar to the requirements of classical business applications. The article by
O. Wolf provides a comprehensive overview of the aspects of \ms{} composition
and discusses the paradigms of orchestration and choreography in detail
\cite{wolf_ms}. Also C. Pahl and B. Lee provide a comprehensive paper that
discusses requirements for the composition of containerized applications in the
cloud \cite{pahl2015containers}.

The requirements for logical composition are in essence very simple. \mss{} must
understand each other. This may sound simple but since logical composition can
be achieved in so many ways, choosing the right solution can be quite a
challenge.

\subsubsection{Deployment of a \ms{} Environment}

The deployment topic consists of all aspects that have to be considered when a
production environment for a \ms{} application is realized. This includes the
continuous integration process that is responsible for the building process of
the application code and for the deployment of executable files and containers
on the production environment.

Deployment also covers the process of choosing an appropriate cloud service
provider to provide the physical infrastructure for the production environment.

\subsubsection{Service Persistence}

The polyglot persistence tenet which is proposed for \mss{} is only partially
usable for \ogs{}. The reason is that a large portion of the game state data
(for example player data) is master data and is essentially used by every
service. With polyglot persistence only one \ms{} would have direct access to
this frequently requested master data. This inevitably leads to a bottle neck.
Therefore other persistence concepts have to be found to cope with master data.

\paragraph{Durability}

Since database technologies are heavily dependent on the underlying storage
technology they are not very well suited for containerization
\cite{cazorla2017db_containers}. The dependence of containers on external
volumes for persistence can introduce a high coupling between container
technology and the database solution, which is undesirable. Also the fluent
nature of containers does not get along very well with durable storage. Although
there are persistent mechanisms to support such durable volumes, data can get
very easily lost in the cloud. This introduces the question of how to actually
integrate a durable database system into a \ms{} application.

\subsubsection{Service Consistency}

One omnipresent problem with distributed applications is data consistency
throughout the shared application state. State consistency in \ogs{} is not
as critical as in the medical, insurance, or financial industry but it is still
desirable. Also it is usually possible to spend real money in an \og{} and
therefore strong consistency is required for parts of the application.

\subsubsection{Service Monitoring}

In order to keep track on the health of a \ms{} application it is mandatory to
have live information on the status of the application. This includes a
visualization of the overall performance of the whole system as well as of
individual services.

This topic also includes notification on events happening in the system like for
example notifications on system failures or regular operational reports. 

\subsection{Networking in \ogs{}}
\label{sub:networking_in_online_games}

Networking is obviously a critical part of \og{} development. It is important
for developers to understand which implications networked functionality has on
game development since a networked game essentially requires a different
application design from a single-player game. 

Networking in video games is a well documented subject. An essential reference
for networking in \ogs{} is the blog by G. Gambetta \cite{gambetta_fast_paced}
about fast-paced \footnote{Fast-paced video games are played in real-time which
means that game events are processed instantaneously and can occur very
frequently (multiple times per second).} multi-player networking.
It is a great reference to understand the low level aspects of networking in games.

Networking for \mss{} and \ogs{} has already been discussed in detail in my
project thesis one \cite{biedermann2015project1} in  section 3.3 Network Game
Technology. But since networking is the most integral part of \ogs{}, networking
requirements always need to be reconsidered according to new findings.


\subsection{Development of a \ms{} Driven Online Game}

A number of tools can be used to simplify the development process of \mss{}.
Many tools in this regard exist and can be of great value if properly used.

This subject is about the evaluation of existing tools according to the
following desirable properties:

\begin{itemize}
  \item Reduce the time needed to introduce new \mss{}
  \item Allow the developer focus on domain logic vs. boilerplate code
  \item Increase the automation level of continuous integration (deployment)
  \item Provide functionality to cope with \mss{} composition (both physical
  and logical composition)
  \item Visualization of operational statistics, communication flow and domain
  specific data like the game model
\end{itemize}

In addition to the evaluation of existing tools also custom concepts and
prototypes are developed in the course of this subject to further simplify the
development process of \ms{} \og{} development.

\subsection{Procedural Content Generation in Online Games}

\glsreset{pcg} \gls{pcg} is a strategy to produce game content procedurally.
\gls{pcg} is an efficient way to reduce the effort needed by designers to
produce all the content needed for a game. Game content produced this way can
lack the desired quality and therefore manual intervention by game designers is
always required. \gls{pcg} has become very popular in recent years
\cite{lee2014procedural} and is therefore a very interesting research topic in
regard to its usability for \og{} development.

\gls{pcg} is usually realized using search-based methods. Especially evolutionary
algorithms perform very well. Some examples are: fractal and noise algorithms,
grammars and L-systems, planning algorithms, and many more. The book Procedural
Content Generation in Games \cite{shaker2014procedural} is a very comprehensive
summary on the topic which is currently widely researched in academia.

\section{Instrumentation}

As a first step to examine the subjects defined in \autoref{sec:subject_pool} a
clear definition of the technical foundation to reproduce the subjects during
the lab research (\autoref{sub:lab_reserach}) has to be made. For this purpose a
comparison of available technology is provided along with a proposal on how to
choose between similar available technologies.

\subsection{MicroNet: A Reference Implementation}

The prototype that has been already mentioned prior \todo{Exact reference}
several times is actually a collection of prototypes representing a reference
implementation on how to realize an \og{} using \ms{}.

The development title of this reference implementation is MicroNet.
MicroNet consists of the following components:

\begin{itemize}
  \item A core framework that provides run-time libraries required to operate a
  \ms{} \og{} application cluster.
  \item A small reference implementation on how to build an \og{} with MicroNet
  \item A service catalogue which contains reference implementation of
  functionality that is commonly uses in \ogs{}.
  \item A tool-set to simplify the development process of an \og{} 
\end{itemize}

\subsubsection{Common Functionality}

The polyglot programming tenet states that a \ms{} is free to choose its
implementation details. For the actual realization of a \ms{} application
this aspect introduces a number of challenges. 

It breaks the DRY (don't repeat yourself) principle if the same piece of
functionality is simply copied among multiple services. To cope with this a
solution is needed that on one hand does respect the DRY principle and on the
other hand provides a mechanism to share similar functionality among services.

One solution for this dilemma is to provide a reference solution that furnishes
a service with common functionality. This reference solution can be used to
build \mss{} which have no special technology requirement quickly.

In the prototype this aspect is tackled using a reference implementation
developed in Java. Java is in theory platform independent and therefore a good
choice to provide a general solution for \mss{} application design and
implementation. \mss{} with extended requirements are free not to rely on the
common solution and participate in the cluster autonomous.

\subsubsection{Application data}

As explained in \autoref{sub:games} an \og{} game operates on data. This data
must be suitable to be persisted and to be sent over a network (these aspects
have been discussed in project thesis 2 \todo{ref}). There is an literally
infinite number of ways on how to treat application data like for example Json,
XML, CVS, Strings, binary data, and HTML just to name a few.

Since \ogs{} are very time critical the fastest and therefore a binary data
format would introduce the least overhead. But since binary serialization
introduces a large amount of additional work during development it can be
considered an optimization\cite{gafferon2017games} and therefore a simpler data
format can be chosen for the prototype. Json is a good compromize in this regard
because it has an acceptable overhead, is human readable and also very popular.

\subsubsection{Networking}

An integral part of \ogs{} is networking. In regards of \mss{} the most common
approaches to networking are RESTful HTTP or Messaging. Messaging offers
several advantages over pure HTTP that are quite useful to make a \ms{}
application robust: Message queues are persistent and in case a service is
unavailable the messages are buffered and delivered as soon as the service
is available again. Furthermore it is possible to combine multiple message
brokers to a mesh to provide fail-over functionality. These aspects make a
message broker the optimal choice to facilitate networking for \ms{} \og{}
applications. Networking has been discussed in detail in project thesis one
\todo{Cite Thesis 1}.

\subsection{Development Environment}

The decision to propose Java as the main programming language for \mss{} allows
to derive a set of assumptions that simplify the development process of \ms{}
applications.

\subsubsection{Java as the Main Programming Language}

The latest Java version 8 introduced functional programming into Java. With
functional programming it is possible to inject code into methods in a
convenient way. They are a perfect match to allow the developer to inject his
own code into a framework. MicroNet makes excessive use of this possibility.

Java enables the utilization of Apache Maven which is a very powerful software
project management and comprehension tool. It allows the describe and automate
the build process of a java applications. It also simplifies the process of
integrating third party java libraries into the application and the majority of
the java libraries are available as maven artifacts.

\subsubsection{Code Generation}

Java version 6 introduced annotation processing in the standard java build
process. This makes it very simple to provide framework functionality closely
related to the actual \og{}. Annotation processing coupled with code generation
are one major driver to facilitate the composition of \mss{}.

\subsubsection{The Eclipse IDE}

When it comes to Java development the Eclipse IDE is very popular. This is
backed by the fact that Eclipse has been around since \todo{Date} and has gone
though a long maturing process. Meanwhile Eclipse is a very robust development
environment not only for Java but also for many other programming languages. 

The way Eclipse is built makes it very easy to extend it. It is basically a set
of plug-ins that together form the IDE. The developer can write it's own
plug-ins to tailor the development process to his needs. 

MicroNet uses the Eclipse plug-in functionality to integrate \ms{} specific
development tools into Eclipse. With mininal effort it is also possible to
extract the MicroNet plug-ins to standalone applications. This however is out of
scope for this thesis.

\subsection{Container Engine}

Containers are a very powerful modern approach to run software. Instead of
installing a software directly on a host it is instead installed in a container.
The container can then run among other containers in a container engine as a
virtual machine. The container engine can be installed on any host that is
supported. This approach greatly increases the portability of software and
allows developers to build software once and run it anywhere as long as the
container engine is available.-

\subsubsection{Docker}

Docker is the most widely used container technology and a de-facto standard
\todo{Citation needed}. The docker ecosystem provides with Dockerhub a central
registry for available Docker containers. This centralized approach makes is
easy to install required dependencies on a system as long as docker is
installed.

\paragraph{Docker Compose}

Docker Compose is a way to describe a complex application consisting of multiple
Docker containers. With this approach it is possible to build or run a complete
\ms{} application with a single command. The structure of docker-compose is
also very well suited for automation. The compose script is in yaml format which
many libraries support.

Docker Compose is also the foundation to deploy a distributed application in
Docker Swarm explained in \autoref{sub:composition_engines}.

\paragraph{Docker for Windows}

As mentioned in \todo{Ref to Windows Requirement} the whole \og{} development
process must be realizable in Windows. Fortunately the Docker for Windows Client
is a very sophisticated tool that feels very close to the original Docker
version on Linux. One problem regarding Docker for Windows is that it does not
run on Windows versions which do not support HyperV. As a work-around developers
can use the Docker Toolbox. Docker Toolbox uses VirtualBox to provide the
same functionality as stand alone Docker but with the price of a little bit more
complexity during setup.

\subsection{Composition Engines}
\label{sub:composition_engines}

The physical \ms{} composition topic is a pure technology evaluation problem.
The selection of technologies for evaluation is the first aspect to consider in
this regard. Among the many available solution the three well known solutions
for \ms{} composition are compared: Kubernetes, Apache Mesos and Docker Swarm.

\subsubsection{Evaluation Criteria}

In order to compare these three composition solutions X properties are defined
and examined for each technology. The categories are chosen according to the
experience of cloud experts in the industry \cite{toll2016cloud_expert_eval}
\cite{lerilli2012cloud_eval_criteria} \cite{voras2011evaluating}.

\paragraph{The Virtualization Layer}

\begin{center}
  \begin{tabular}{ | p{4.2cm} | p{4.2cm} | p{4.2cm} | }
    \hline
    \textbf{Kubernetes}&\textbf{Apache Mesos}&\textbf{Docker Swarm}\\\hline
    4 & 5 & 6 \\
    \hline
  \end{tabular}
\end{center}

\paragraph{The Storage Layer}

\begin{center}
  \begin{tabular}{ | p{4.2cm} | p{4.2cm} | p{4.2cm} | }
    \hline
    \textbf{Kubernetes}&\textbf{Apache Mesos}&\textbf{Docker Swarm}\\\hline
    4 & 5 & 6 \\
    \hline
  \end{tabular}
\end{center}

\paragraph{The Management Layer}

\begin{center}
  \begin{tabular}{ | p{4.2cm} | p{4.2cm} | p{4.2cm} | }
    \hline
    \textbf{Kubernetes}&\textbf{Apache Mesos}&\textbf{Docker Swarm}\\\hline
    4 & 5 & 6 \\
    \hline
  \end{tabular}
\end{center}

\paragraph{Initial Provisioning Time}

\begin{center}
  \begin{tabular}{ | p{4.2cm} | p{4.2cm} | p{4.2cm} | }
    \hline
    \textbf{Kubernetes}&\textbf{Apache Mesos}&\textbf{Docker Swarm}\\\hline
    4 & 5 & 6 \\
    \hline
  \end{tabular}
\end{center}

\paragraph{Ease of Use/ Ease of Setup}

\begin{center}
  \begin{tabular}{ | p{4.2cm} | p{4.2cm} | p{4.2cm} | }
    \hline
    \textbf{Kubernetes}&\textbf{Apache Mesos}&\textbf{Docker Swarm}\\\hline
    4 & 5 & 6 \\
    \hline
  \end{tabular}
\end{center}

\paragraph{Reduce of Technology Complexity}

\begin{center}
  \begin{tabular}{ | p{4.2cm} | p{4.2cm} | p{4.2cm} | }
    \hline
    \textbf{Kubernetes}&\textbf{Apache Mesos}&\textbf{Docker Swarm}\\\hline
    4 & 5 & 6 \\
    \hline
  \end{tabular}
\end{center}

\paragraph{Customization}

\begin{center}
  \begin{tabular}{ | p{4.2cm} | p{4.2cm} | p{4.2cm} | }
    \hline
    \textbf{Kubernetes}&\textbf{Apache Mesos}&\textbf{Docker Swarm}\\\hline
    4 & 5 & 6 \\
    \hline
  \end{tabular}
\end{center}

\paragraph{Infrastructure Monitoring and Reporting Ability}

\begin{center}
  \begin{tabular}{ | p{4.2cm} | p{4.2cm} | p{4.2cm} | }
    \hline
    \textbf{Kubernetes}&\textbf{Apache Mesos}&\textbf{Docker Swarm}\\\hline
    4 & 5 & 6 \\
    \hline
  \end{tabular}
\end{center}

\paragraph{Customization}

\begin{center}
  \begin{tabular}{ | p{4.2cm} | p{4.2cm} | p{4.2cm} | }
    \hline
    \textbf{Kubernetes}&\textbf{Apache Mesos}&\textbf{Docker Swarm}\\\hline
    4 & 5 & 6 \\
    \hline
  \end{tabular}
\end{center}

\paragraph{Exclusivity}

If the relationship goes sour, how easy is it extract your legacy data?

\begin{center}
  \begin{tabular}{ | p{4.2cm} | p{4.2cm} | p{4.2cm} | }
    \hline
    \textbf{Kubernetes}&\textbf{Apache Mesos}&\textbf{Docker Swarm}\\\hline
    4 & 5 & 6 \\
    \hline
  \end{tabular}
\end{center}


\paragraph{Exclusivity}

If the relationship goes sour, how easy is it extract your legacy data?

\begin{center}
  \begin{tabular}{ | p{4.2cm} | p{4.2cm} | p{4.2cm} | }
    \hline
    \textbf{Kubernetes}&\textbf{Apache Mesos}&\textbf{Docker Swarm}\\\hline
    4 & 5 & 6 \\
    \hline
  \end{tabular}
\end{center}

\paragraph{Application Description Format}

If the relationship goes sour, how easy is it extract your legacy data?

\begin{center}
  \begin{tabular}{ | p{4.2cm} | p{4.2cm} | p{4.2cm} | }
    \hline
    \textbf{Kubernetes}&\textbf{Apache Mesos}&\textbf{Docker Swarm}\\\hline
    4 & 5 & 6 \\
    \hline
  \end{tabular}
\end{center}
 
\paragraph{Supported Eco Systems}

If the relationship goes sour, how easy is it extract your legacy data?

\begin{center}
  \begin{tabular}{ | p{4.2cm} | p{4.2cm} | p{4.2cm} | }
    \hline
    \textbf{Kubernetes}&\textbf{Apache Mesos}&\textbf{Docker Swarm}\\\hline
    4 & 5 & 6 \\
    \hline
  \end{tabular}
\end{center}

\paragraph{Scalability}

If the relationship goes sour, how easy is it extract your legacy data?

\begin{center}
  \begin{tabular}{ | p{4.2cm} | p{4.2cm} | p{4.2cm} | }
    \hline
    \textbf{Kubernetes}&\textbf{Apache Mesos}&\textbf{Docker Swarm}\\\hline
    4 & 5 & 6 \\
    \hline
  \end{tabular}
\end{center}

\paragraph{Liecence}

If the relationship goes sour, how easy is it extract your legacy data?

\begin{center}
  \begin{tabular}{ | p{4.2cm} | p{4.2cm} | p{4.2cm} | }
    \hline
    \textbf{Kubernetes}&\textbf{Apache Mesos}&\textbf{Docker Swarm}\\\hline
    4 & 5 & 6 \\
    \hline
  \end{tabular}
\end{center}
 
\subsection{Database Solution}
\subsection{Source and Artifact Control}
\subsection{Cloud Service Provider}

\subsection{Available Technology}

\subsubsection{Cloud Service Provider}

MicroNet is designed to run in the cloud. It is therefore mandatory to evaluate
the major cloud service infrastructure providers according to their usability
for online games.

One problem in regards to these service providers is payment. Although they
provide a free tier to test their infrastructure, a credit card has to be
pledged as security. In the case the limit is overdrawn a fee will be charged.
This risk cannot be taken in this thesis. 

\paragraph{Google Cloud Gaming Solutions}

Google offers a solution to deploy rich online games in the cloud. It basically
a setup for the existing google cloud infrastructure with additional
functionality to to provide build in matchmaking matchmaking. For this purpose
it provides an API that is available in a number of programming languages.

The MicroNet service that provides the matchmaking functionality has to be
altered if the Google solution for matchmaking wants to be used. 

To operate a game driven by MicroNet in the Google Cloud the Google App Engine
can be used to run the services of the MicroNet framework and the Google Compute
Engine can be used to run the game simulation servers. ActiveMQ that is used for
messaging can be used with Google Cloud by using Bitnami.

\paragraph{Amazon Game Lift}

Amazon GameLift is very similar to Google Gaming Solutions. It also provides a
matchmaking solution and the option to communicate with the game backend running
in the Amazon cloud. An advantage of the amazon solution is that it provides an
API for all the major game engines like Unity3D or Unreal Engine 4. Amazon also
provides its own game engine with the name Lumberyard.

\paragraph{A Dedicated Server}

The most general way to deploy the MicroNet framework is on a dedicated server
hardware. This could be either a cluster of bare metal servers with a sufficient
networking capabilities or a set of virtual servers running somewhere in the
cloud. Either way the result is a set of hosts running linux and have a public
ip address.

For this thesis a virtual server at the HSR will be the test setup. This
infrastructure is also available at the major cloud service providers.





\subsection{Game Engine}

\subsection{The Spacegame Prototype}


\section{Procedures}

Procedures describe the actions that have to be taken in order to evaluate the
research problems defined in \autoref{sub:problems}.

These actions can separated into the three categories: Lab research, field
research, and statistical analysis.

\subsection{Lab Research}
\label{sub:lab_reserach}

A major goal of the lab research is to examine the \ms{} infrastructure
requirements listed in \autoref{sub:infrastructure}. The following table
illustrates how the instrumentation described in \autoref{sub:instrumentation}
is used to examine the infrastructure requirements. 

\begin{center}
  \begin{tabular}{ l | l | l | l | l | l | }
    &\textbf{Composition}&\textbf{Deployment}&\textbf{Persistence}&\textbf{Consistency}&\textbf{Monitoring}\\\hline
    \textbf{MicroNet}\\\hline
    \textbf{Development Environment}\\\hline
    \textbf{Container Engine}\\\hline
    \textbf{Cloud Service Providers}\\\hline
    \textbf{Composition Engines}\\\hline
    \textbf{Message Broker}\\\hline
    \textbf{Database}\\\hline
    \textbf{Source Control}\\\hline
    \textbf{ Dedicated Server}\\\hline
  \end{tabular}
\end{center}


\subsection{Field Research}

\subsection{Statistical Analysis}

\subsection{Method Selection for Research Problems}

Based on the Infrastructure described in \autoref{sub:infrastructure} concrete
strategies can now be defined on how to investigate the subjects defined in
\autoref{sec:subject_pool}.

\subsubsection{}

For the examination of the subjects it is therefor necessary to access each
subject with in a specifically suited method. The research problems can be
summarized into four problem domains which allows to choose the the optimal
research method for each domain individually.

\subsubsection{Composition}
Expert Opinion on Model Approach

\subsubsection{Deployment}
Evaluation of 3rd party libraries (DevOps work)


\subsubsection{Development}
Expert Opinion

A one click solution for developers to start with online game
  	  development. The development process should feel as closely as possible to
  	  offline game development.

\subsubsection{Operations}
Play Tests






 
















