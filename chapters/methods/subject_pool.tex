\section{Subject Pool}
\label{sec:subject_pool}

The research problems listed in \autoref{sub:problems} which describe subjects
of this thesis are very variegated. To gain an overview over all the activities
that have to be conducted to examine the research problems concrete research
subjects can be formed. These subjects are defined to be well suited for
individual examination.

\subsection{Challenges Introduced by \mss{}}
\label{sub:ms_challenges}

One main aspect to verify the \ogs{} and \mss{} hypothesis
(\autoref{sub:hypothesis}) is to find \ms{} friendly solutions for the major
challenges which \og{} development introduces. A solution can be considered
\ms{} friendly if it suffices all seven \ms{}tenets.

\subsubsection{State Isolation}

Most noticeable \ms{} state isolation according to the IDEAL tenet represents a
major challenge. This is because a game is always state-full\footnote{A game
always has a state like for example the positions of all chess pieces on a chess
board is the state of the chess game.}. Since \mss{} are always stateless a
solution must be found on how to make the game state available to \mss{}.

This challenge is common in computer science disciplines and is referred to as
session management. In the case of \ogs{} sessions of players have to be
tracked but also the sessions of the game itself (e.g. one game round).

\subsubsection{Model Access}

The state isolation challenge inevitably leads to another challenge namely the
game model access. Since the game state is shared among multiple \mss{} a
flexible approach is needed to provide access to session data to the services.

\subsection{\ms{} Infrastructure}
\label{sub:infrastructure}

The \ms{} infrastructure subject is a collection of all topics that have to be
considered when developing and operating a \ms{} environment. The content of
this subject is not restricted to the \ogs{} and can be applied to any domain.

\subsubsection{Requirements for \ms{} Composition}

The examination of the requirements for \ms{} composition can be grouped into
logical and physical composition requirements as explained in
\autoref{subsub:composition}.

The requirements of physical composition demand that \mss{} need a way to find
each other despite where they are deployed. This requirement is tightly coupled
with physical characteristics of the target environment. The requirements for
physical composition of \ms{} \og{} environments are very similar to the
requirements of classical business applications. The article of O.Wolf provides
a very good overview of the aspects of \ms{} composition and discusses the
paradigms orchestration and choreography in detail \cite{wolf_ms}. Also C.Pahl
and B.Lee provide a comprehensive paper that discusses requirements to compose
containerized applications in the cloud \cite{pahl2015containers}.

The requirements for logical composition are in essence very simple. \mss{} must
understand each other. This may sound simple but since logical composition can
be achieved in so many ways, choosing the right solution can be quite a
challenge.

\subsubsection{Deployment of a \ms{} Environment}

The deployment topic consists of all aspects that have to be considered when a
production environment for a \ms{} application is realized. This includes the
continuous integration process that is responsible to build the application code
and deploy the executable files to the production environment.

Deployment also covers the process of choosing an appropriate cloud service
provider to provide the physical infrastructure for the production environment.

\subsubsection{Service Persistence}

The polyglot persistence tenet which is proposed for \mss{} is only partially
usable for \ogs{}. The reason is that a large portion of the game state data
(for example player data) is essentially used by every service. With polyglot
persistence only one \ms{} would have direct access to this frequently requested
data. This inevitably leads to a bottle neck. Therefore other persistence
concepts have to be found to cope with this base data.

\paragraph{Durability}

Since database technologies are heavily dependent of the underlying storage
technology they are not very well suited for containerization
\cite{cazorla2017db_containers}. The dependence of containers on external
volumes for persistence it can introduce a high coupling between the container
technology and the database solution which is undesirable. Also the fluent
nature of containers does not get along with durable storage very well. Although
there are mechanisms to persist such durable volumes, data can get lost in the
cloud very easily. This introduces the question on how to actually integrate a
durable database system into a \ms{} application.

\subsubsection{Service Consistency}

One omnipresent problem with distributed applications is data consistency
throughout the application. For \ogs{} consistency is not as critical as like in for example
the medical industry but still undesirable. Also usually it is possible to spend
real money in an \og{} and therefore strong consistency is required for parts of
the application. 

\subsubsection{Service Monitoring}

In order to keep track on the health of a \ms{} application it is mandatory to
have live information on the status of the application. This includes a
visualization of the overall performance of the whole system and also for
individual services.

This topic also includes notification on events happening in the system like for
example notifications on system failure or regular operational reports. 

\subsection{Networking in \ogs{}}
\label{sub:networking_in_online_games}

Networking is obviously a critical part of \og{} development. It is important
that developers understand which implications networked functionality has on
game development since a networked games requires a essentially different
application design than single-player games. 

Networking in video games is a well documented subject. An essential reference
for networking in \ogs{} is the blog of G.Gambetta\cite{gambetta_fast_paced}
about fast-paced multi-player. It is a great reference to understand the low
level aspects of networking in games.

Networking for \mss{} and \og{} has already been discussed in detail in project
thesis one in \todo{3.3 Network Game Technology}. But since networking is the
most integral part of \ogs{}, networking requirements always need to be
reconsidered according to new findings about the problem domains.


\subsection{Development of a \ms{} Driven Online Game}

A number of tools can be used to simplify the development process of \ms{}. Many
tools in this regard exist an can be of great value if properly used.

This subject is about the evaluation of existing tools according to the
following desirable properties:

\begin{itemize}
  \item Reduce the time needed to introduce new \mss{}
  \item Allow the developer focus on domain logic and not boilerplate code
  \item Increasing the automation level of continuous integration (deployment)
  \item Provide functionality to cope with \mss{} composition (both physical
  and logical composition)
  \item Visualization of operational statistics, communication flows and domain
  specific data like the game model.
\end{itemize}

In addition to the evaluation of existing tools also custom concepts and
prototypes are developed in the course of this subject to further simplify the
development process of \ms{} \og{} development.

\subsection{Procedural Content Generation in Online Games}

Procedural Content Generation (PCG) a strategy to produce game content
procedurally. PCG is an efficient way to reduce the effort needed by designers
to produce all the content needed for a game. Game content produced this way can
lack the desired quality and therefore manual intervention by game designers is
always required. PCG has become very popular in recent years
\cite{lee2014procedural} and is therefor a very interesting research topic in
regards to its usability for \og{} development.

PCG is usually realized using search-based methods. Especially evolutionary
algorithms perform very well. Some examples are: Fractal and noise algorithms,
grammars and L-systems, Planning algorithms, and many more. The book Procedural
Content Generation in Games \cite{shaker2014procedural} is a very comprehensive
summary on the topic which is currently widely researched in academia.
