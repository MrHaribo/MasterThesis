\section{Subject Pool}
\label{sec:subject_pool}

The research problems that this thesis aims to solve are very variegated. To
gain an overview over all the activities that have to be conducted in the course
of this thesis the research problems can be collated to concrete subjects. These
subjects can then be examined individually.

\subsection{\ms{} Tenets in Relation to Online Games}

This is the major subject of this thesis and will therefore receive a lot of
attention. The research goal of this subject is to research every \ms{} tenet
according to its requirements in regards to \ogs{}. This is done in the form of
a requirements analysis of every tenet and a proposal on how to suffice them.
Examples for these solutions are implemented in the prototype and serve as a
demonstration. \todo{Ref to Result Section}

\subsection{Challenges Introduced by \mss{}}
\label{sub:ms_challenges}

It is challenging to suffice all \ms{} tenets for \og{} development. Most
noticeable the IDEAL tenet on specifically isolated state represents the biggest
challenge. This is because a game is always stateful. For example the location
of all chess pieces on a chess board is the state of the chess game. Since
\mss{} are always stateless the full game state must be sent with every game
message to provide true service confinement. In a simple game like chess this
might be possible but in a rich \og{} this is just no feasible. 

This challenge also exists in other computer science disciplines and is called
session management. In the case of \ogs{} sessions of players have to be
tracked but also the sessions of the game itself (e.g. one round of chess).

This challenge leads to another challenge namely game model access. Since the
game state is shared among multiple \mss{} a flexible approach is needed to
provide access to session data to the services.

The remaining challenges have already been discussed in the previous two project
theses. \todo{list and reference them}

\subsection{\ms{} Infrastructure}
\label{sub:infrastructure}

The \ms{} infrastructure subject is a collection of all topics that have to be
considered when operating a \ms{} environment. The content of this subject is
not restricted to the \ogs{} and can be applied to any domain.

\subsubsection{Requirements for \ms{} Composition}

The examination of the requirements for \ms{} composition can be grouped into
logical and physical composition (explained in \autoref{subsub:composition})
requirements.

The requirements for physical composition of \ms{} \og{} environments are very
similar to the requirements of classical business applications. The article of
O.Wolf provides a very good overview of the aspects of \ms{} composition and
discusses the paradigms orchestration and choreography in detail \cite{wolf_ms}.
Also C.Pahl and B.Lee provide a comprehensive paper that discusses requirements
to compose containerized applications in the cloud \cite{pahl2015containers}.

The requirements for logical composition are in essence very simple. Physical
composition offers a way for \mss{} to find each other. Logical composition is
on top of physical composition responsible that the services understand each
other.

\subsubsection{Deployment of a \ms{} Environment}

The deployment topic contains all aspects that have to be considered when a
production environment for a \ms{} application is realized. This includes the
continuous integration process that is responsible to build the application code
and deploy the executable files to the production environment.

Deployment also covers the aspect of choosing an appropriate cloud service
provider to that provides the physical infrastructure for the production
environment.

\subsubsection{Service Persistence}

The polyglot persistence approach proposed for \mss{} is only partially usable.
This is because a large portion of the state data (for example player data)
is essentially used by every service. With polyglot persistence only one
\ms{} would have direct access to this frequently requested data. This
inevitably leads to a bottle neck.

\paragraph{Durability}

In \ms{} environments databases are often deployed as \mss{} themselves.
Therefor it is in fact possible to loose the data stored in the containers. To
cope with this a durable solution is needed that adds a layer of redundancy to
guarantee that no data is lost. 

\subsubsection{Service Consistency}

One omnipresent problem with distribution is data consistency throughout the
application. For \ogs{} consistency is not as critical as like in for example
the medical industry but still undesirable. Also usually it is possible to spend
real money in an \og{} and therefore strong consistency is required for parts of
the application. 

\subsubsection{Service Monitoring}

In order to keep track on the health of a \ms{} application it is mandatory to
have live information on the status of the application. This includes a
visualization of the overall performance of the whole system and also for
individual services.

This topic also includes notification on events happening in the system like for
example notifications on system failure or regular operational reports. 

\subsection{Networking in \ogs{}}

Networking is obviously a critical part of \og{} development. it is important
that developers understand which implications the networked functionality has on
the game. Networking in video games is a well documented subject. An essential
reference for networking in \ogs{} is the blog of
G.Gambetta\cite{gambetta_fast_paced} about fast-paced multi-player. It is a
great reference to understand the low level aspects of networking in games.


\subsection{Development of a \ms{} Driven Online Game}

A number of tools in addition to the basic IDE can simplify the development
process of \ms{}. Many tools in this regard exist an can be of great value if
properly used. 

The following properties are desirable when selecting or developing tools:

\begin{itemize}
  \item Reduce the time needed to introduce new \mss{}
  \item Allow the developer focus on domain logic and not boilerplate code
  \item Increasing the automation level of continuous integration (deployment)
  \item Provide functionality to cope with \mss{} composition (both physical
  and logical composition)
  \item Visualization of operational statistics, communication flows and domain
  specific data like the game model.
\end{itemize}

\subsection{Procedural Content Generation in Online Games}

Procedural Content Generation (PCG) a strategy to produce game content
procedurally. PCG is an efficient way to reduce the effort needed by designers
to produce all the content needed for a game. Game content produced this way can
lack the desired quality and therefore manual intervention by game designers is
always required. PCG has become very popular in recent years
\cite{lee2014procedural} and is therefor a very interesting research topic in
regards to its usability for \og{} development.

PCG is usually realized using search-based methods. Especially evolutionary
algorithms perform very well. Some examples are: Fractal and noise algorithms,
grammars and L-systems, Planning algorithms, and many more. The book Procedural
Content Generation in Games \cite{shaker2014procedural} is a very comprehensive
summary on the topic which is currently widely researched in academia.
