\section{Statistical Analysis}

The results of the previous semesters are to this point only of hypothetical
nature. The main goal this semester is therefore to validate the findings up to
now.\\

Validation for the run-time aspect of the solution:

\begin{enumerate}
  \item To prove that the solution holds in regard to run-time is necessary to
  give proper validation feedback for the combination of Microservices and
  Online Games. It must be shown the running framework is able to scale at
  run-time. In a first instance this is done by performing lab-research in the
  form of small scale play test with some elected test players.
  \item One more sophisticated approach is to perform TAR on the run-time
  validation. This would mean to test the MicroNet with the alpha version of the
  Spacegame prototype in a long running tests that is available for the public. 
\end{enumerate}

Validation of the development-time aspect of the solution:

\begin{enumerate}
  \item It must be proven that the proposed development solution is actually
  usable and as simple as proposed.
\end{enumerate}

This validation research is made according to design science methods and
teminology.