\section{Instrumentation}
\label{sub:instrumentation}

The first step to examine the subjects defined in \autoref{sec:subject_pool} is
to define a clear technical foundation used to reproduce the subjects during lab
research (\autoref{sub:lab_reserach}). This section contains an analysis of
available technology along with a proposal on how to choose between similar
available technologies.

\subsection{MicroNet: A Reference Implementation}

MicroNet is the working title of the prototype that serves as a reference
implementation of how to realize an \og{} using \mss{}. The implementation of a
prototype was already an issue of the last two semester theses and this thesis
aims at finalizing and releasing MicroNet.

MicroNet consists of the following components:

\begin{itemize}
  \item A core framework that provides run-time libraries required to operate a
  \ms{} \og{} application cluster
  \item A small reference implementation on how to build an \og{} with MicroNet
  \item A service catalogue which contains reference implementation of
  functionality that is commonly used in \ogs{}.
  \item A tool-set to simplify the development process of an \og{} 
\end{itemize}

\subsection{Java as Programming Language}

Java was chosen as the primarily used programming language at the beginning of
project thesis one because it is the promoted programming language at \gls{hsr}.
In general the polyglot programming tenet states that a \ms{} is free to choose
its implementation details according to its requirements. For the realization of
a \ms{} application this tenet tends to break the \gls{dry} principle if the
same piece of functionality is simply copied among multiple services.

The strong commitment to Java as a programming language offers the
advantage that functionality can be shared among services using Java features
and therefore allows to quickly develop an application while preventing
duplicated code.

The downside of a prominent programming language is that in order to still allow
polyglot programming additional functionality has to be provided to allow
foreign services to communicate with the application. The storage paradigm of
foreign services is based on the polyglot persistence tenet and therefore not a
concern in this regard.

\subsubsection{Java 8 Lambdas}

Java version 8 introduced functional programming into Java. With
functional programming it is possible to inject code into methods in a
convenient way. They are a perfect match to allow the developer to inject his
own code into a framework. MicroNet makes excessive use of this possibility.

\subsubsection{Apache Maven}

Java enables the utilization of Apache Maven which is a very powerful software
project management and a comprehension tool. It permits to describe and automate
the build process of a Java application. It also simplifies the process of
integrating third party Java libraries into the application, and the majority of
Java libraries are available as Maven artifacts.

\subsubsection{Annotation Processing and Code Generation}

Java version 6 introduced annotation processing in the standard java build
process. It is a very powerful concept which can be used to generate code at
compile time according to annotated classes and methods. Annotation processing
paired with code generation can spare the developer a lot of boilerplate code.

\subsubsection{The Eclipse IDE}

For Java development the Eclipse \gls{ide} is very popular. This is backed by
the fact that Eclipse has been around since 2004 and has gone through a long
maturing process since then. Meanwhile Eclipse is a very robust development
environment not only for Java but also for many other programming languages.

The way Eclipse is built makes it very easy to extend it. It is basically a set
of plug-ins that together form the \gls{ide}. A developer can develop his own
plug-ins to tailor the development process of an application according to his
needs.

MicroNet uses the Eclipse plug-in functionality to integrate \ms{} specific
development tools into Eclipse.

\subsection{Application Data Format}

As explained in \autoref{sub:games} an \og{} game always has a state. The state
is represented as a set of dynamic data in the game model. The format of this
data must be suitable for both persistence and network communications.
Respecting both requirements makes the application code reusable and cleaner.

There is an literally infinite number of ways of how to format application data.
Some examples are: \gls{json}, \gls{xml}, \gls{cvs}, Strings, binary data, and
\gls{html}.

Since fast-faced \ogs{} are very performance critical the fastest - and
therefore a binary data format - would introduce the least overhead. But since
binary serialization includes a large amount of additional work during
development it can be considered an
optimization\footnote{\cite{gafferon2017games} shows many tricks of how to
reduce \og{} network bandwith using binary serialization.}, and therefore a
simpler data format will be used for MicroNet.
\gls{json} is a good choice in this regard because it has an acceptable
overhead, is human readable and also very popular.

\subsection{Networking Technologies}

As explained in \autoref{sub:networking_in_online_games} networking is an
integral part of every \og{}. The \ms{} tenet on fine-grained interfaces and
single responsibility states that most common approaches to networking for \ms{}
applications are RESTful \gls{http} solutions. But as explained in section 3.3
Network Game Technology of project thesis one \cite{biedermann2015project1} for
\ogs{} using a message broker offers several advantages over pure \gls{http}
while still allowing the exposition of the \gls{api} as RESTful \gls{http}.

\subsubsection{Message Broker}

Message brokers are quite useful to strengthen a \ms{} application's robustness:
They provide persistent message queues, so if no service is listening to a
specific queue the undelivered messages are buffered and delivered as soon as a
service becomes available to process the messages. This allows the realization
of a self repairing application. Furthermore it is possible to combine multiple
message brokers to form a mesh of brokers that provides resilience to message
broker fail-overs. The advantages of message brokers make them the optimal
choice to facilitate networking for \ms{} \og{} applications.

ActiveMQ is used as a message broker within a MicroNet application to provide
communication capabilities to \mss{}. For security reasons two separate message
brokers are used. One is exposed to the public Internet and accepts user
connections. A second broker is responsible to forward messages to the
application internally. A reverse proxy (API Gateway) is used to interconnect
the two. The message broker evaluation has been conducted in project thesis one
\cite{biedermann2015project1} in section 4.5 Communication Mechanisms in a \ms{}
Architecture.

\subsection{Container Engine}

Containers are a very powerful modern approach to run software. Instead of
installing a software directly on a host it is instead installed in a container.
The container can then be started among other containers in a container engine
as a virtual machine. The only dependency that containers introduce is the
container engine which must be installed on the host. The host can be anything :
A developer laptop, a dedicated server, or a virtual server in the cloud.
This approach greatly increases the portability of software and allows
developers to build software once and run it anywhere as long as the container
engine is available.

\subsubsection{Docker}

Docker is the most widely used container technology and a de-facto standard
since most container engines support docker containers. The docker ecosystem
provides a central registry for available Docker containers to be used by
developers called Dockerhub. This centralized approach makes it easy to download
required dependencies on a system as long as the Docker engine is installed.

\paragraph{Docker Compose}

Docker Compose is a way to describe a complex application consisting of multiple
Docker containers. It is used to build or run a complete \ms{} application with
a single command. The structure of docker-compose is also very well suited for
automation since it is formatted in the well known .yaml format which is
supported by many libraries.

Docker Compose is also the foundation to deploy a distributed application in
Docker Swarm, which is explained in \autoref{sub:composition_engines}.

\paragraph{Docker for Windows}

As mentioned in \autoref{sub:delimitations} the whole \og{} development process
must be realizable on Windows. Fortunately the Docker for Windows Client is a
very sophisticated tool that feels very close to the original Docker version on
Linux. One problem regarding Docker for Windows is that it does not run on
Windows versions which do not support HyperV (Windows Home Edition). As a
work-around developers may use the Docker Toolbox. Docker Toolbox uses
VirtualBox to provide the same functionality like as the stand alone Docker but
with the price of a little bit more complexity during setup.

\subsection{Cloud Service Providers}

Currently it has become very popular for companies to use the infrastructure of
the major cloud service providers to deploy their infrastructure
\cite{roberts2016aws}, \cite{pronschinske2015turbine}. The providers offer a
stability and performance that cannot be achieved by a proprietary operated
server center \cite{gai2012towards}.

Because of this trend it is mandatory to consider the largest players in this
market as a reference on how to deploy modern distributed applications. Therefore the
usability of a technology is partially influenced by cloud service present
support. 

Amazon Cloud (AWS) and Google Cloud Platform (GCP) are the major cloud service
providers which are considered in this thesis. It has to be mentioned that due
to the limitation mentioned in \autoref{sub:zero_buget} these cloud service
providers cannot be examined in practice. However the comprehensive
documentation that these providers provide allows us to gain an insight into
their technology.

\newpage
\subsection{Composition Engines}
\label{sub:composition_engines}

The physical \ms{} composition topic is a pure technology evaluation problem.
In this regard three well known technologies among the many available solutions 
for \ms{} composition are compared: Kubernetes, Apache Mesos and Docker Swarm.

\subsubsection{Evaluation Criteria}

In order to compare these three composition solutions twelve properties
(P1-P12) are examined for each technology. The categories are chosen according
to the experience of cloud experts in the industry
\cite{toll2016cloud_expert_eval}, \cite{lerilli2012cloud_eval_criteria}, and
\cite{voras2011evaluating}.


\paragraph{P1 - Virtualization Layer}

Virtualization technology is used by the composition engine to operate a
cluster of \mss{}. Virtualization technology is tightly coupled with the
different platforms the technology supports.

\begin{center}
  \begin{tabular}{ | p{4.5cm} | p{4.5cm} | p{4.5cm} | }
    \hline
    \textbf{Kubernetes}&\textbf{Apache Mesos}&\textbf{Docker Swarm}\\\hline
    -Pods are the smallest units that make up an application deployment. A pod
    is an autonomous unit of the application stack. Pods can confine one or more
    containers.
    
    -Supports AWS, GCP (using the Google Compute Engine - GCE), Microsoft Azure,
    Joyent, OpenStack, VMWare, bare metal and localhost deployments.
    
    -Minikube is a way to run and test Kubernetes localhost deployments. It
    supports the VM drivers: virtualbox, vmwarefusion, kvm, and xhyve.
    
    -CoreOS is a way to install Kubernetes on bare metal.&
    
    -Mesos consists of a master daemon that manages agent daemons running on
    each cluster node. Each daemon can run containers as tasks. 
    
    -Mesos supports Docker containers but also offers a proprietary solution in
    the form of Mesos Containerizers.
    
    -With Mesosphere DC/OS (the datacenter operating system) bare metal
    installations are possible. &
    
    -Multiple Docker Daemons on different physical machines provide a cluster
    of nodes.
    
    -Docker Daemons are available for Windows, MacOS, and Linux. Within Linux
    Docker is realized with native Linux Containers. The native Windows
    installation uses HyperV.
    
    -Docker Toolbox provides a way to use Docker with virtualbox.
    
    -Docker Daemons are also available for AWS and Azure.\\
    \hline
  \end{tabular}
\end{center}


\newpage
\paragraph{P2 - Storage Layer}

Since containers only live non-permanently in cloud memory they have no
dedicated physical storage attached to them. Therefore the composition engine
must provide a way for containers to persist data that exceed the lifetime of a
container. The storage layer also provides the possibility to share data among
containers.

Many DBMS systems provide a containerized solution. The storage layer provided
by the composition engine is then used by DBMS containers to persist the data on
disk.

\begin{center}
  \begin{tabular}{ | p{4.5cm} | p{4.5cm} | p{4.5cm} | }
    \hline
    \textbf{Kubernetes}&\textbf{Apache Mesos}&\textbf{Docker Swarm}\\\hline
    -Kubernetes uses an approach called Persistent Volumes to provide a common
    \gls{api} to manage the storage layer. 
    
    -Supports the storage technologies of: GCP, AWS, Azure, NFS, Vsphere,
    VMware and more.&

    -It is suggested to use accessible storage outside the containers for
    example a native database (NoSQL/SQL).
    
    -Offers Persistent Volumes comparable to the local filesystem. They
    are still accessible after a task terminates and other tasks can consume the
    persisted data during execution.
    
    -The Docker Containerizer supports the Docker Volume plug-in. & 
     
    -Host-based persistence allows multiple containers to access a shared
    volume on the host bypassing the union file-system. Data corruption can be
    the consequence.
    
    -Implicit Per-Container Storage is a storage sandbox per container on
    the host. When a container is removed, so is it's a storage sandbox.
    
    -Explicit Shared Storage (Data Volumes) mounts an explicit location on the
    host within one or more containers allowing shared read-write access to
    a common directory.
    
    -Data Volumes can be provided through a shared file-system like Ceth,
    GlusterFS, or NFS and exposed through a consistent name-space.\\
    \hline
  \end{tabular}
\end{center}

\newpage
\paragraph{P3 - Management Layer}

The management layer is responsible to provide an interface to start, stop and
monitor an application stack or a single container in the virtualization layer.

\begin{center}
  \begin{tabular}{ | p{4.5cm} | p{4.5cm} | p{4.5cm} | }
  	\hline
    \textbf{Kubernetes}&\textbf{Apache Mesos}&\textbf{Docker Swarm}\\\hline
    -kubectl is the main command line tool to manage a Kubernetes cluster.
    
    -kubefed is a command line tool to combine and federate multiple
    clusters.

	-minikube is a command line tool to administrate localhost clusters.
    
    -The Web UI (Dashboard) offers a graphical user interface to manage nodes,
    deployments, and volumes in the cluster.
    
    -The kompose tool can be used to translate a docker-compose application to
    a Kubernetes application. &
    
    -Offers many specific command line tools for DevOps and Developers.
    
    -Supports several \gls{api} client libraries to control the cluster.
    
    -No graphical user interface to manage the cluster is available out of the
    box. &
    
    -docker-compose command line tool  can be used to describe and build
    containerized applications.
    
    -docker stack command line tool is used to manage an application stack.
    
    -No graphical user interface to manage the cluster is available out of the
    box.\\
    
    \hline
  \end{tabular}
\end{center}

\paragraph{P4 - Provisioning Time}

The initial provisioning time and the time to scale up and down the application
can be an indicator of the simplicity of a technology. The provisioning time can
be compared by deploying a demo application with all composition engines.
These measurements include the time needed to understand the documentation to a
sufficient extent as well as the time needed to install the necessary software
to deploy the demo application.

\begin{center}
  \begin{tabular}{ | p{4.5cm} | p{4.5cm} | p{4.5cm} | }
    \hline
    \textbf{Kubernetes}&\textbf{Apache Mesos}&\textbf{Docker Swarm}\\\hline
    $<$ 3 days & Not Evaluated & $<$ 1 day \\
    \hline
  \end{tabular}
\end{center}

\newpage
\paragraph{P5 - Ease of Use}

Ease of use impression of simplicity that the deployment of the demo application
makes.

\begin{center}
  \begin{tabular}{ | p{4.5cm} | p{4.5cm} | p{4.5cm} | }
    \hline
    \textbf{Kubernetes}&\textbf{Apache Mesos}&\textbf{Docker Swarm}\\\hline
    -The dashboard provides a way to easily understand a Kubernetes
    cluster.
    
    -On Windows the setup of minikube is cumbersome.
    
    -Kubernetes proprietary scripts are complicated and hard to understand. &
    
    -Mesos is hard to set up because it has many dependencies like Zookeeper
    or the different scheduling plug-ins.
    
    -Ease of operation and monitoring was not evaluated for Mesos. & 
    
    -When Native Docker is understood by the developer Docker Swarm is very
    easy to understand.
    
    -Docker Swarm is very slim, so a full picture of the technology can be
    gained very quickly.
    
    -Although no native dashboard exists the management of Docker with the
    command line is very comfortable.\\
    \hline
  \end{tabular}
\end{center}

\paragraph{P6 - Reduction of Technology Complexity}

The reduction of technology complexity is considered in regard to networking
among containers, persistent storage among containers and the effort needed to
integrate dependencies in an application.

\begin{center}
  \begin{tabular}{ | p{4.5cm} | p{4.5cm} | p{4.5cm} | }
    \hline
    \textbf{Kubernetes}&\textbf{Apache Mesos}&\textbf{Docker Swarm}\\\hline
    -The Web UI provides services and discovery, workloads, and persistent
    volumes as convenient ways to abstract dependent technologies.
    
    -Dependencies can be abstracted using dedicated containers for dependencies.
    
    -Containers are shared among developers using private registries
    or Google, Amazon or Azure Container Registries.&
    
    -Mesos agents have to be configured according to networking and storage.
    
    -Docker containers can be used to abstract dependencies. & 
    
    -The Docker Engine makes it possible to define overlay networks. This aspect
    of Docker is not trivial and requires knowledge in regards to virtual
    networking technologies.
    
    -Persistent storage and networking can be defined via the Docker and
    docker-compose files but knowledge about this storage technology is
    required.
    
    -Dependency abstraction is provided by Docker-hub, the central docker image
    repository. Docker hub is not restricted to Docker-Swarm.\\
    \hline
  \end{tabular}
\end{center}

\newpage
\paragraph{P7 - Infrastructure Monitoring and Reporting Ability}

Infrastructure monitoring and reporting abilities include the ability to observe
a running \ms{} application stack. This involves alerts on abnormal situations
as well as the visual representation of statistics and metrics.

\begin{center}
  \begin{tabular}{ | p{4.5cm} | p{4.5cm} | p{4.5cm} | }
    \hline
    \textbf{Kubernetes}&\textbf{Apache Mesos}&\textbf{Docker Swarm}\\\hline
    -Heapster as a cluster-wide aggregator for monitoring and event data.
    
    -cAdvisor can be used as an open source container resource usage and
    performance analysis tool.
    
    -A Grafana setup with InfluxDB is a very popular combination for
    monitoring.&
    
    -Mesos nodes report a set of statistics and metrics that
    enable cluster operators to monitor resource usage and detect abnormal
    situations early enough.
    
    -Observability metrics in the following categories are available:
    Resources, Masters, Agents, Frameworks, Tasks, Messages, Event queues
    and more.&
    
    -Does not provide a built-in solution for reporting.
    
    -Depends on the underlying infrastructure for reporting.
    
    -Grafana in addition to cAdvisor and Prometheus can be used to implement an
    open source reporting and monitoring solution.\\
    \hline
  \end{tabular}
\end{center}

\paragraph{P8 - Customization}

Customization includes the ability to provide new functionality or change
existing behaviour.

\begin{center}
  \begin{tabular}{ | p{4.5cm} | p{4.5cm} | p{4.5cm} | }
    \hline
    \textbf{Kubernetes}&\textbf{Apache Mesos}&\textbf{Docker Swarm}\\\hline
    -Custom \gls{api} types allow the developer to define new functionality and add it
    to the Kubernetes \gls{api}.
    
    -Add-ons can be developed to extend the functionality of Kubernetes.
    
    -Support for custom networking plug-ins are currently in alpha.& 
    
    -Is used to add custom framework schedulers in C++, Go, Haskell, Java,
    Python and Scala.
    
    -RENDLER provides an example framework implementation realized with Mesos.& 
    
    -Extensions can be made by using the Docker Engine managed plug-in system.
    
    -Provides SDKs for the Docker Engine \gls{api} in Python and GO. These
    \glspl{api} are tested by Docker maintainers.
    
    -Third party SDKs are available in: C/C++, C\#, Java, and more.\\
    \hline
  \end{tabular}
\end{center}

\paragraph{P9 - Exclusivity}

Exclusivity involves the question of how easy is it extract your legacy data,
upon a technology change or upon deprecation of a technology.

\begin{center}
  \begin{tabular}{ | p{4.5cm} | p{4.5cm} | p{4.5cm} | }
    \hline
    \textbf{Kubernetes}&\textbf{Apache Mesos}&\textbf{Docker Swarm}\\\hline
    -Proprietary solution prevents direct re-usage of an application stack. & 
    
    -Mesos Containerizers are not portable but Docker Containerizers are. &
    
    -The full application stack can be translated to another technology very
    easily due to the base technology nature of Docker.\\
    \hline
  \end{tabular}
\end{center}

\paragraph{P10 - Application Description Format}

This is the format that is used to describe an application stack.

\begin{center}
  \begin{tabular}{ | p{4.5cm} | p{4.5cm} | p{4.5cm} | }
    \hline
    \textbf{Kubernetes}&\textbf{Apache Mesos}&\textbf{Docker Swarm}\\\hline
    -Kubernetes deployment description files in .yml format. & 
    
    -Task Group \gls{api} files (proprietary). & 
    
    -docker-compose file in .yml format. \\
    \hline
  \end{tabular}
\end{center}

\paragraph{P11 - Scalability}

This is the ability of the technology to scale an application stack according to
performance requirements.

\begin{center}
  \begin{tabular}{ | p{4.5cm} | p{4.5cm} | p{4.5cm} | }
    \hline
    \textbf{Kubernetes}&\textbf{Apache Mesos}&\textbf{Docker Swarm}\\\hline
    -Horizontal Pod auto-scaling allows Kubernetes to automatically scale the
    number of pods based on observed CPU utilization. & 
    
    -CD/OS provides auto-scaling using Marathon. & 
    
    -Provides no auto-scaling mechanism, but manual scaling is very simple and
    can be automated with little effort.
    \\
    \hline
  \end{tabular}
\end{center}

\paragraph{P12 - Liecence}

This includes the licence the technology is released under.

\begin{center}
  \begin{tabular}{ | p{4.5cm} | p{4.5cm} | p{4.5cm} | }
    \hline
    \textbf{Kubernetes}&\textbf{Apache Mesos}&\textbf{Docker Swarm}\\\hline
    -Apache License 2.0 & 
    
    -Apache License 2.0 & 
    
    -Community Editon (free) 
    
    -Enterprise Edition with a per node / per year fee\\
    \hline
  \end{tabular}
\end{center}

\subsection{Database Solution}
\label{sub:database_solutions}

The polyglot persistence tenet states that every \ms{} should use the data base
solution that is best suited for its requirements. As described in
\autoref{sub:infrastructure} it is possible to offer databases to service
containers as database containers. For production environments however this
approach may be unstable \cite{cazorla2017db_containers}. Containerized
databases can still be useful for localhost deployments in order to test
functionality, but for production environments a dedicated database system
proves to be more stable.

Session management is a big concern in \ogs{} as highlighted in
\autoref{sub:ms_challenges} and in \autoref{sub:infrastructure}. Providing a
database solution for session management can be facilitated by using a
distributed cache such as Redis which is a very container friendly solution.
Also NoSQL databases offer data expiration functionality which can be used
for session management too.

MicroNet uses PostgreSQL as an easy method to containerize a relational database
and Couchbase as a NoSQL database to store session information. Couchbase is also
used to make the game model as described in \autoref{sub:games} highly
available.

\subsection{Source and Artifact Control}

All the source code of this thesis is available on github
\cite{micronet2017doku}. The github page also contains the user manual for
MicroNet in \gls{html} format. MicroNet is also Available on Maven Central to
allow easy integration into Java projects. All the the dependencies like the
message broker or the databases are available on Docker Hub as Docker
containers.

\subsection{Cloud Service Provider Gaming Solutions}

The major cloud service providers Amazon and Google provide dedicated solutions
to host \og{} environments. These solutions are deeply integrated into the
respective cloud technology of the provider and offer additional functionality
in order to ease the deployment of \ogs{}. Due to the restrictions mention in
\autoref{sub:zero_buget} these solutions have only briefly been examined.

\subsubsection{Google Cloud Gaming Solutions}

Google's solution to deploy rich \ogs{} in the cloud is basically a setup
for the existing Google cloud infrastructure with additional functionality to
provide matchmaking. For this purpose the Google Compute Engine \gls{api} is used
which is available in a number of programming languages like Java, C\#, GO,
Phthon, NodeJS, and others. This covers the majority of game engines and
allows them to use the Google Gaming Solution.

\subsubsection{Amazon Game Lift}

Amazon GameLift is very similar to Google Gaming Solutions as it also provides
a matchmaking solution. This is realized by a back-end framework running in the
Amazon cloud. This back-end is customizable by the developer. 

GameLift also offers an \gls{api} which allows game simulation servers to
incorporate with the backend. This \gls{api} is available for all the major game
engines like Unity3D or Unreal Engine 4. Amazon also provides its own game
engine under the name Lumberyard.

\subsection{Dedicated Server}

The simplest way to deploy a \ms{} application is a dedicated server. Using only
a single dedicated server for the application defies the advantages of \ms{}
applications to be robust and scalable. This approach has the same problem as a
monolithic application.

But all the evaluated composition engines detach the physical environment from
the application and therefore the development and deployment process is basically
the same on a single machine or a cluster of nodes. Because of this reason a
single machine is sufficient as a target environment for lab research.

A dedicated server in this respect could be either a bare metal server or a
virtual server in the cloud. In both cases the environment is very similar for
the developer: a host running Linux which is reachable over the Internet. For
this thesis a virtual server at HSR will be used for lab research.

\subsection{Game Engine}

The game engine used to evaluate game engine integration in a \ms{} application
is Unity3D. Unity3D is a very popular game engine and is permanently being
improved. An evaluation of different game engines has been conducted in project
thesis one \cite{biedermann2015project1} in section 6.1.1 Game Engines.

\subsection{The Spacegame Prototype}

The Spacegame prototype is a project that I am working on in my free time. Since
my master studies have had a span over four semesters and I only wrote three
theses I found some time during the additional semester to build a game prototype using
MicroNet. The Spacegame prototype is an example of a more sophisticated game
than the example game included in MicroNet.

