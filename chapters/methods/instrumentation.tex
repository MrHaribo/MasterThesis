\section{Instrumentation}

As a first step to examine the subjects defined in \autoref{sec:subject_pool} a
clear definition of the technical foundation to reproduce the subjects during
the lab research (\autoref{sub:lab_reserach}) has to be made. For this purpose a
comparison of available technology is provided along with a proposal on how to
choose between similar available technologies.

\subsection{MicroNet: A Reference Implementation}

The prototype that has been already mentioned prior \todo{Exact reference}
several times is actually a collection of prototypes representing a reference
implementation on how to realize an \og{} using \ms{}.

The development title of this reference implementation is MicroNet.
MicroNet consists of the following components:

\begin{itemize}
  \item A core framework that provides run-time libraries required to operate a
  \ms{} \og{} application cluster.
  \item A small reference implementation on how to build an \og{} with MicroNet
  \item A service catalogue which contains reference implementation of
  functionality that is commonly uses in \ogs{}.
  \item A tool-set to simplify the development process of an \og{} 
\end{itemize}

\subsubsection{Common Functionality}

The polyglot programming tenet states that a \ms{} is free to choose its
implementation details. For the actual realization of a \ms{} application
this aspect introduces a number of challenges. 

It breaks the DRY (don't repeat yourself) principle if the same piece of
functionality is simply copied among multiple services. To cope with this a
solution is needed that on one hand does respect the DRY principle and on the
other hand provides a mechanism to share similar functionality among services.

One solution for this dilemma is to provide a reference solution that furnishes
a service with common functionality. This reference solution can be used to
build \mss{} which have no special technology requirement quickly.

In the prototype this aspect is tackled using a reference implementation
developed in Java. Java is in theory platform independent and therefore a good
choice to provide a general solution for \mss{} application design and
implementation. \mss{} with extended requirements are free not to rely on the
common solution and participate in the cluster autonomous.

\subsubsection{Application data}

As explained in \autoref{sub:games} an \og{} game operates on data. This data
must be suitable to be persisted and to be sent over a network (these aspects
have been discussed in project thesis 2 \todo{ref}). There is an literally
infinite number of ways on how to treat application data like for example Json,
XML, CVS, Strings, binary data, and HTML just to name a few.

Since \ogs{} are very time critical the fastest and therefore a binary data
format would introduce the least overhead. But since binary serialization
introduces a large amount of additional work during development it can be
considered an optimization\cite{gafferon2017games} and therefore a simpler data
format can be chosen for the prototype. Json is a good compromize in this regard
because it has an acceptable overhead, is human readable and also very popular.

\subsubsection{Networking}

An integral part of \ogs{} is networking. In regards of \mss{} the most common
approaches to networking are RESTful HTTP or Messaging. Messaging offers
several advantages over pure HTTP that are quite useful to make a \ms{}
application robust: Message queues are persistent and in case a service is
unavailable the messages are buffered and delivered as soon as the service
is available again. Furthermore it is possible to combine multiple message
brokers to a mesh to provide fail-over functionality. These aspects make a
message broker the optimal choice to facilitate networking for \ms{} \og{}
applications. Networking has been discussed in detail in project thesis one
\todo{Cite Thesis 1}.

\subsection{Development Environment}

The decision to propose Java as the main programming language for \mss{} allows
to derive a set of assumptions that simplify the development process of \ms{}
applications.

\subsubsection{Java as the Main Programming Language}

The latest Java version 8 introduced functional programming into Java. With
functional programming it is possible to inject code into methods in a
convenient way. They are a perfect match to allow the developer to inject his
own code into a framework. MicroNet makes excessive use of this possibility.

Java enables the utilization of Apache Maven which is a very powerful software
project management and comprehension tool. It allows the describe and automate
the build process of a java applications. It also simplifies the process of
integrating third party java libraries into the application and the majority of
the java libraries are available as maven artifacts.

\subsubsection{Code Generation}

Java version 6 introduced annotation processing in the standard java build
process. This makes it very simple to provide framework functionality closely
related to the actual \og{}. Annotation processing coupled with code generation
are one major driver to facilitate the composition of \mss{}.

\subsubsection{The Eclipse IDE}

When it comes to Java development the Eclipse IDE is very popular. This is
backed by the fact that Eclipse has been around since \todo{Date} and has gone
though a long maturing process. Meanwhile Eclipse is a very robust development
environment not only for Java but also for many other programming languages. 

The way Eclipse is built makes it very easy to extend it. It is basically a set
of plug-ins that together form the IDE. The developer can write it's own
plug-ins to tailor the development process to his needs. 

MicroNet uses the Eclipse plug-in functionality to integrate \ms{} specific
development tools into Eclipse. With mininal effort it is also possible to
extract the MicroNet plug-ins to standalone applications. This however is out of
scope for this thesis.

\subsection{Container Engine}

Containers are a very powerful modern approach to run software. Instead of
installing a software directly on a host it is instead installed in a container.
The container can then run among other containers in a container engine as a
virtual machine. The container engine can be installed on any host that is
supported. This approach greatly increases the portability of software and
allows developers to build software once and run it anywhere as long as the
container engine is available.-

\subsubsection{Docker}

Docker is the most widely used container technology and a de-facto standard
\todo{Citation needed}. The docker ecosystem provides with Dockerhub a central
registry for available Docker containers. This centralized approach makes is
easy to install required dependencies on a system as long as docker is
installed.

\paragraph{Docker Compose}

Docker Compose is a way to describe a complex application consisting of multiple
Docker containers. With this approach it is possible to build or run a complete
\ms{} application with a single command. The structure of docker-compose is
also very well suited for automation. The compose script is in yaml format which
many libraries support.

Docker Compose is also the foundation to deploy a distributed application in
Docker Swarm explained in \autoref{sub:composition_engines}.

\paragraph{Docker for Windows}

As mentioned in \todo{Ref to Windows Requirement} the whole \og{} development
process must be realizable in Windows. Fortunately the Docker for Windows Client
is a very sophisticated tool that feels very close to the original Docker
version on Linux. One problem regarding Docker for Windows is that it does not
run on Windows versions which do not support HyperV. As a work-around developers
can use the Docker Toolbox. Docker Toolbox uses VirtualBox to provide the
same functionality as stand alone Docker but with the price of a little bit more
complexity during setup.

\subsection{Composition Engines}
\label{sub:composition_engines}

The physical \ms{} composition topic is a pure technology evaluation problem.
The selection of technologies for evaluation is the first aspect to consider in
this regard. Among the many available solution the three well known solutions
for \ms{} composition are compared: Kubernetes, Apache Mesos and Docker Swarm.

\subsubsection{Evaluation Criteria}

In order to compare these three composition solutions X properties are defined
and examined for each technology. The categories are chosen according to the
experience of cloud experts in the industry \cite{toll2016cloud_expert_eval}
\cite{lerilli2012cloud_eval_criteria} \cite{voras2011evaluating}.

\paragraph{The Virtualization Layer}

\begin{center}
  \begin{tabular}{ | p{4.2cm} | p{4.2cm} | p{4.2cm} | }
    \hline
    \textbf{Kubernetes}&\textbf{Apache Mesos}&\textbf{Docker Swarm}\\\hline
    4 & 5 & 6 \\
    \hline
  \end{tabular}
\end{center}

\paragraph{The Storage Layer}

\begin{center}
  \begin{tabular}{ | p{4.2cm} | p{4.2cm} | p{4.2cm} | }
    \hline
    \textbf{Kubernetes}&\textbf{Apache Mesos}&\textbf{Docker Swarm}\\\hline
    4 & 5 & 6 \\
    \hline
  \end{tabular}
\end{center}

\paragraph{The Management Layer}

\begin{center}
  \begin{tabular}{ | p{4.2cm} | p{4.2cm} | p{4.2cm} | }
    \hline
    \textbf{Kubernetes}&\textbf{Apache Mesos}&\textbf{Docker Swarm}\\\hline
    4 & 5 & 6 \\
    \hline
  \end{tabular}
\end{center}

\paragraph{Initial Provisioning Time}

\begin{center}
  \begin{tabular}{ | p{4.2cm} | p{4.2cm} | p{4.2cm} | }
    \hline
    \textbf{Kubernetes}&\textbf{Apache Mesos}&\textbf{Docker Swarm}\\\hline
    4 & 5 & 6 \\
    \hline
  \end{tabular}
\end{center}

\paragraph{Ease of Use/ Ease of Setup}

\begin{center}
  \begin{tabular}{ | p{4.2cm} | p{4.2cm} | p{4.2cm} | }
    \hline
    \textbf{Kubernetes}&\textbf{Apache Mesos}&\textbf{Docker Swarm}\\\hline
    4 & 5 & 6 \\
    \hline
  \end{tabular}
\end{center}

\paragraph{Reduce of Technology Complexity}

\begin{center}
  \begin{tabular}{ | p{4.2cm} | p{4.2cm} | p{4.2cm} | }
    \hline
    \textbf{Kubernetes}&\textbf{Apache Mesos}&\textbf{Docker Swarm}\\\hline
    4 & 5 & 6 \\
    \hline
  \end{tabular}
\end{center}

\paragraph{Customization}

\begin{center}
  \begin{tabular}{ | p{4.2cm} | p{4.2cm} | p{4.2cm} | }
    \hline
    \textbf{Kubernetes}&\textbf{Apache Mesos}&\textbf{Docker Swarm}\\\hline
    4 & 5 & 6 \\
    \hline
  \end{tabular}
\end{center}

\paragraph{Infrastructure Monitoring and Reporting Ability}

\begin{center}
  \begin{tabular}{ | p{4.2cm} | p{4.2cm} | p{4.2cm} | }
    \hline
    \textbf{Kubernetes}&\textbf{Apache Mesos}&\textbf{Docker Swarm}\\\hline
    4 & 5 & 6 \\
    \hline
  \end{tabular}
\end{center}

\paragraph{Customization}

\begin{center}
  \begin{tabular}{ | p{4.2cm} | p{4.2cm} | p{4.2cm} | }
    \hline
    \textbf{Kubernetes}&\textbf{Apache Mesos}&\textbf{Docker Swarm}\\\hline
    4 & 5 & 6 \\
    \hline
  \end{tabular}
\end{center}

\paragraph{Exclusivity}

If the relationship goes sour, how easy is it extract your legacy data?

\begin{center}
  \begin{tabular}{ | p{4.2cm} | p{4.2cm} | p{4.2cm} | }
    \hline
    \textbf{Kubernetes}&\textbf{Apache Mesos}&\textbf{Docker Swarm}\\\hline
    4 & 5 & 6 \\
    \hline
  \end{tabular}
\end{center}


\paragraph{Exclusivity}

If the relationship goes sour, how easy is it extract your legacy data?

\begin{center}
  \begin{tabular}{ | p{4.2cm} | p{4.2cm} | p{4.2cm} | }
    \hline
    \textbf{Kubernetes}&\textbf{Apache Mesos}&\textbf{Docker Swarm}\\\hline
    4 & 5 & 6 \\
    \hline
  \end{tabular}
\end{center}

\paragraph{Application Description Format}

If the relationship goes sour, how easy is it extract your legacy data?

\begin{center}
  \begin{tabular}{ | p{4.2cm} | p{4.2cm} | p{4.2cm} | }
    \hline
    \textbf{Kubernetes}&\textbf{Apache Mesos}&\textbf{Docker Swarm}\\\hline
    4 & 5 & 6 \\
    \hline
  \end{tabular}
\end{center}
 
\paragraph{Supported Eco Systems}

If the relationship goes sour, how easy is it extract your legacy data?

\begin{center}
  \begin{tabular}{ | p{4.2cm} | p{4.2cm} | p{4.2cm} | }
    \hline
    \textbf{Kubernetes}&\textbf{Apache Mesos}&\textbf{Docker Swarm}\\\hline
    4 & 5 & 6 \\
    \hline
  \end{tabular}
\end{center}

\paragraph{Scalability}

If the relationship goes sour, how easy is it extract your legacy data?

\begin{center}
  \begin{tabular}{ | p{4.2cm} | p{4.2cm} | p{4.2cm} | }
    \hline
    \textbf{Kubernetes}&\textbf{Apache Mesos}&\textbf{Docker Swarm}\\\hline
    4 & 5 & 6 \\
    \hline
  \end{tabular}
\end{center}

\paragraph{Liecence}

If the relationship goes sour, how easy is it extract your legacy data?

\begin{center}
  \begin{tabular}{ | p{4.2cm} | p{4.2cm} | p{4.2cm} | }
    \hline
    \textbf{Kubernetes}&\textbf{Apache Mesos}&\textbf{Docker Swarm}\\\hline
    4 & 5 & 6 \\
    \hline
  \end{tabular}
\end{center}
 
\subsection{Database Solution}
\subsection{Source and Artifact Control}
\subsection{Cloud Service Provider}

\subsection{Available Technology}

\subsubsection{Cloud Service Provider}

MicroNet is designed to run in the cloud. It is therefore mandatory to evaluate
the major cloud service infrastructure providers according to their usability
for online games.

One problem in regards to these service providers is payment. Although they
provide a free tier to test their infrastructure, a credit card has to be
pledged as security. In the case the limit is overdrawn a fee will be charged.
This risk cannot be taken in this thesis. 

\paragraph{Google Cloud Gaming Solutions}

Google offers a solution to deploy rich online games in the cloud. It basically
a setup for the existing google cloud infrastructure with additional
functionality to to provide build in matchmaking matchmaking. For this purpose
it provides an API that is available in a number of programming languages.

The MicroNet service that provides the matchmaking functionality has to be
altered if the Google solution for matchmaking wants to be used. 

To operate a game driven by MicroNet in the Google Cloud the Google App Engine
can be used to run the services of the MicroNet framework and the Google Compute
Engine can be used to run the game simulation servers. ActiveMQ that is used for
messaging can be used with Google Cloud by using Bitnami.

\paragraph{Amazon Game Lift}

Amazon GameLift is very similar to Google Gaming Solutions. It also provides a
matchmaking solution and the option to communicate with the game backend running
in the Amazon cloud. An advantage of the amazon solution is that it provides an
API for all the major game engines like Unity3D or Unreal Engine 4. Amazon also
provides its own game engine with the name Lumberyard.

\paragraph{A Dedicated Server}

The most general way to deploy the MicroNet framework is on a dedicated server
hardware. This could be either a cluster of bare metal servers with a sufficient
networking capabilities or a set of virtual servers running somewhere in the
cloud. Either way the result is a set of hosts running linux and have a public
ip address.

For this thesis a virtual server at the HSR will be the test setup. This
infrastructure is also available at the major cloud service providers.





\subsection{Game Engine}

\subsection{The Spacegame Prototype}

