\section{Summary of Findings}

The research motivation for this thesis is \ms{} development, especially
\ms{} composition and deployment. In these two areas I gathered a lot of
knowledge during this thesis and in this section I want to share the findings
that i find most relevant. These findings represent my attempt to answer the
knowledge questions in \autoref{sub:problems}.

\subsection{Usability of a \ms{} for Online Games}

\noindent
\textbf{Can an Online Game run in a \ms{} environment?}

Yes, it is possible to run an \og{} in a \ms{} environment. I has already been
done in the industry (\cite{pronschinske2015turbine}) and MicroNet is an example
on how to accomplish this.\\

\noindent
\textbf{Is a Microservice influenced architecture suitable to design an Online
Game?}

The design of a \mss{} application heavily influences the way \ogs{} are
developed. This type of architecture forces the developer to split up the game
logic into small shard of \ms{} ``size''. This enforces that large domain chunks
which have high cohesion (like for example all player related functionality)
have to be split into smaller chunks. It is sometimes difficult to achieve
needed functionality this way. A good example of this aspect is the shop service
which served is explained in detail in \todo{autoref thesis 2}.\\

\noindent
\textbf{Is the added complexity during development tolerable?}

\ms{} indeed  adds an overhead during development. This effort pays off later
since scaling and extending of the application are simplified greatly. The added
effort occurs mainly at the beginning of a game project for example to set up
the continuous integration work-flow. For large teams (AAA companies) this
overhead can be neglected since a DevOps team can positioned to cope with these
boiler plate tasks beforehand and during the project.

For small teams (independent developers) however the overhead can be quite
troublesome. Once the development environment for \ms{}is set up the effort is
equivalent to regular \og{} development. MicroNet is aimed to fill exactly this
gab and make it easier to start with \ms{} driven \og{} development.\\

\noindent
\textbf{Do \mss{} have enough performance to drive a fast-paced \ogs{}?}

This question is very hard to answer in the context of this thesis. To give
sophisticated assumptions about the performance of a \ms{} application a  
full-fledged test game and a large enough group of testers is needed. The
Spacegame which I developed in my free time is not compete enough to fulfill
this role. During this semester i had basically no time at all to improve the
stability of the Spacegame to a level suited for TAR. Also the acquisition of
testers can be quite challenging since it involves a lot of community work
which is out of scope for this thesis.

\subsection{Deployment and Operation}

\noindent
\textbf{Which steps are necessary to bring a \ms{} driven \og{} �from code to
the cloud�?}

From code to could includes all steps necessary to build and deploy the complete
\ms{} \og{} application. It is assumed the the application code is fully
functional and tested. It is also assumed that the target environment is set-up
correctly meaning the Docker engine is running and all dependencies mentioned in
\todo{ref dependencies} are installed. The steps are:

\begin{itemize}
  \item Build the executable of each \ms{} locally
  \item Build a Docker container each containing a service executable along with
  all required dependencies locally
  \item Test the application locally
  \item Upload the complete application code to the target environment using git
  or ftp
  \item Build the application on the target environment using the same
  configuration as locally
  \item Optionally start dependencies which are not integrated in the
  application
  \item Start the application using the docker CLI 
\end{itemize}

\noindent
\textbf{How can the deployment process be automated?}

Maven and Docker are great to reduce and automate the number of steps performed
by the developer to deploy a \ms{} application. The launch utility of MicroNet
combines both techniques by automating the generation and execution of the
docker-compose respectively the master .pom file which further reduces the
number of manual steps.

The MicroNet Launch utility provides shortcuts for all deployment steps
locally. The process of uploading and starting the application on the target
environment varies greatly according to the chosen solution. My recommendation
is therefore to execute the necessary console commands by hand on the server
terminal. The MicroNet documentation provides a guide on how to do
this\cite{micronet2017doku}.\\

\noindent
\textbf{How well do \ogs{} run in a cluster-environment?}

As already mentioned in \todo{autoref no Tar possible} it was difficult to
gather real world performance data on \ms{} driven \ogs{}. But despite that the
few play tests using the Spacegame prototype that where conducted showed that
MicroNet works in general. 

Several short (10minutes) play tests over the Internet with
testers in Berlin, Vienna, Switzerland and Liechtenstein showed that the
Networking concept of MicroNet is robust enough for a unreliable Network like
the Internet.

One extended test of the Spacegame in a Local Area Network (LAN) with 6
simultanious player also went very smoothly. But since this semester I had no
time at all to work on the Spacegame, no additional play tests have been
possible during this semester.\\

\noindent
\textbf{How does a modern container engine assist in deploying and operating a
\ms{} driven application?}

A big part of effort during this thesis went to the evaluation of composition
solutions of containerized application. Containers are a great tool to
completely omit the need of installing dependencies. During development of an
application you will often encounter the situation: ``Aha, there is a
container available for this dependency, very nice!''

This general availability of dependencies as containers makes it very simple to
test new technology without going through a complex installing process. Although
the dynamic nature of containers are sometimes a burden when it comes to overlay
networks and shared volumes. There aspects of containerization are quite complex
even for Docker which i consider reasonably simple.

Kubernetes and Mesos despite being very powerful solution but are very complex
and setting up an application can be quite the challenge. My recommendation is
to use Kubernetes if enough resources are available to set up such a system from
scratch or if a cloud service provider like Amazon or Google is used to deploy
the application. These providers natively support Kubernetes applications.\\

\noindent
\textbf{Which tools exist that help the developer with deployment?}

I did not find many tools that assist in the deployment process of \ms{}
applications. I believe this is because of the very diverse natures of \ms{}
applications. Since each service is developed individually no tools are needed
to for the development of the overall application.

MicroNet challenges this practice by providing a set of tools to exactly
simplify this aspect. 

\subsection{\ms{} Composition}

\noindent
\textbf{What degree of composition is necessary for a \ms{} application
to work?}

The application data and communication are the only concepts needed to logically
compose a \ms{} application. The data concept is needed to make provide a common
data access layer to facilitate session management and ``Stammdaten''
functionality. Communication concept is needed to define the format and the
semantics of message transfers. Both of these aspects are covered by MicroNet.

For physical composition it is enough to rely on a composition engine like
Docker Swarm. Since these engines are very good at abstracting physical
environment circumstances I always recommend using a composition engine.\\

\noindent
\textbf{Which of the common paradigms orchestration and choreography is suited better
for \ms{} composition?}

For physical composition orchestration is very popular since most composition
engines (all three evaluated) are orchestration solution. Im not aware of a
composition engine that uses choreography for physical composition. 

For logical composition, choreography is a good solution. Choreography is a very
abstract concept. One way to think of it that a \ms{} does never know the state
of its environment (e.g. if another service is available). This design requires
the intense use of time-outs, retries, and roll-backs. In my opinion
choreography is hard to get right but provides the best results in
application robustnes and scaling.\\

\noindent
\textbf{Which middle-ware is most useful in providing \ms{} composition?}

In regard the three composition engines I consider Docker Swarm as the simplest
to learn but with a shortage of convenience features. Kubernetes offers alot of
build in functionality to the price of a much more complicated application
development. Apache Mesos although not examined in detail appears to be
somewhere between the two. It offers many features but i found the documentation
not as comprehensive as the other two.\\

\noindent
\textbf{How can multiple game features be developed in parallel in different
\mss{} while preserving the overall application behaviour?}

\mss{} must first of all define a clean fine grained interface through its API.
MicroNet then provides assistance in offering the API to other \mss{}
\todo{bzw}, the developers developing it. This allows to rely on functionality
provided by other \ms{} teams without knowing any implementation details and
use the networked API functionality like it where local function calls. 

The same goes for persistence in regards on how to access data in a \ms{}. This
data access must be universal to allow a \ms{} to operate on the ``Stammdaten''.
In reality this means that data is stored in databases or caches. Since a
service is usually designed so multiple instances of it can be deployed
(exceptions \todo{ref}) programmers cannot store data in memory (e.g. member
variables). This would lead to a state-ful service which is poison for \ms{}
applications.\\

\noindent
\textbf{What problems are introduced by coupling game features loosely?}

The loose coupling of game features forces the developer to think about service
separation and service interfaces very early. A change in this architecture
critical decisions can propagate over multiple services.

One critical point of coupling in a \og{} application is player specific data.
Since in every game this data is the integral part of the game and is basically
related to the whole game application it is difficult to split such a cetral
domain. One solution to cope with this is to use a mechanism like a shared cache
or a NoSQL database to provide the player specific data to the services. This
solution provides the necessary performance and does not break the \ms{}
principles.\\

\noindent
\textbf{How does the consistency vs. performance trade-off impact \ogs{}?}

Fast-paced games are very performance critical. This regard has been discussed
in detail in \todo{ref thesis 1}. In this thesis the performance issues have
been simplified by separating the performance requirements of the real-time
aspects from the ones of the game from the back-end functionality which is not
that performance critical. The performance of an application can always be
improved by profiling it and improving its bottlenecks.

The consistence requirements of \ogs{} can also be separated into two groups.
Strong consistency for important transaction (when they include real money
transaction, or significant time effort by the player) and non-important
transactions (moving the avatar in a MMO).

There is always a trade-off between the two \todo{citation needed} and MicroNet
is more on the performance then on the persistence side. In my opinion players
will tolerate minor discrepancies in the game in general works solid.\\

\noindent
\textbf{What protocols help to reduce service coupling?}

Usually \ms{} provide a RESUful HTTP API to maximize availability. In the case
of REST I agree with the common opinion that it is the way to go for \mss{}. In
regard to way REST is realized i prefer a message broker over a web service.
This is because the message broker offers much more flexibility with the message
format and provides functionality like persistence and topics which a web server
does not. Also REST it is possible to implement REST only partially and still
achieve similar results. The divergence to pure rest has been discussed in
\todo{ref thesis 1}.\\

\subsection{\ms{} Development}

\noindent
\textbf{What separates \ms{} development from regular development?}

As this thesis indicates, \ms{} composition is the big addition to regular
monolithic application development. The major challenge is to develop
functionality in a way which is suitable for \ms{}. This is for example to keep
a service stateless or to find errors that span over multiple services. 

\noindent
\textbf{Which tools help with \ms{} development?}

The MicroNet tools in this regard a specifically designed to help with \ms{}
development. 

Additionally the Docker Tools for Eclipse help to quickly manage
the docker engine by providing wizards to build and run containers. 

SourceTree is a very comprehensive git tool and greatly helps to keep track of
\ms{} application version control since they often span over multiple git
repositories.

