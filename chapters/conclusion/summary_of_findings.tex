\section{Summary of Findings}
\label{sub:summary_of_findings}

The research motivation for this thesis is to understand \ms{} development and
especially \ms{} composition and deployment. In these two areas I have gathered
a lot of knowledge during the process of this thesis and in this section I want
to share the most relevant findings. The findings are presented as
answers to the research questions listed in \autoref{sub:problems}.

\subsection{Usability of a \mss{} for Online Games}
\label{sub:usability_ms_og}

\noindent
\textbf{Can an \og{} run in a \ms{} environment?}

Yes, it is possible to run an \og{} in a \ms{} environment. It has already been
done in the industry \cite{pronschinske2015turbine} and MicroNet also
provides a complete example of how to run an \og{} using \mss{}.\\

\noindent
\textbf{Is a \ms{} influenced architecture suitable to design an \og{}?}

The design of an \og{} \ms{} application heavily influences the way \ogs{} are
classically developed. The \ms{} driven design forces the developer to split up
the game logic into small shards of \ms{} ``size'', which is not a simple task.
Sometimes this means that large domain chunks - which in a monolith would
have a high cohesion - have to be split up, which complicates the application
logic due to distribution circumstances. It is difficult to make a distributed system
behave exactly the same way as a monolith. The canonical example in this regard
is the shop service according to which this problem was explained in detail in
project thesis two \cite{biedermann2016project2} in section 5.8.3 Shop Service.

Therefore the distribution aspect complicates development but also results in
much more scalable and extendable design. For an \og{} that is planned to live
for a long time a \ms{} influenced architecture is therefore very well suited.\\

\newpage
\noindent \textbf{ Is the added complexity which \ms{} development
introduces tolerable in comparison to traditional \og{} development?}

\mss{} indeed  add an overhead during development. However this effort pays off
since scaling and extending of the application are greatly simplified in the
later course of the project. The added effort mainly occurs at the beginning of
a game project for example to set up the continuous integration work flow. For
large teams (AAA companies) this overhead can be neglected since a DevOps team
can positioned to cope with these boiler plate tasks beforehand and during the
project.

For small teams (independent developers) however the overhead can be quite
troublesome. Once the development environment for \ms{} has been set up the
effort during development is mostly equivalent to regular \og{} development but
the provisioning time can have quite an impact. MicroNet is aimed at filling
exactly this gap and making it easier to start with \ms{} driven \og{}
development.

The development of MicroNet itself is a good example for front up effort in
\ms{} development. By the end of this thesis the framework has matured and
meanwhile the development of game functionality has been noticeable simplified
compared to the early versions of MicroNet.\\

The list below identifies tasks which in my opinion introduce a noticeable
overhead:
\begin{itemize}
  \item Setting up the development environment on developer machines
  \item Defining the API of the application early
  \item Implementing and scaling the chosen database solutions
  \item Designing robust cloud affine solutions for consistency requirements
  \item Tracking issues which propagate over multiple services
  \item Testing of distributed application behaviour
\end{itemize}

\noindent \textbf{Do \ms{} applications provide enough performance to
drive a fast-paced \og{}?}

This question is very difficult to answer in the context of this thesis. To give
sophisticated assumptions about the performance of a \ms{} application a
fully-fledged test game and a large enough group of testers are needed. The
Spacegame prototype is not mature enough at this point to fulfill this role.
During this semester there was basically no time at all to improve the stability
of the Spacegame to a level suited for \gls{tar}. Also the acquisition of
testers can be quite challenging since it involves a lot of community work which
is out of scope for this thesis.

\newpage
\subsection{Deployment and Operation}

\noindent
\textbf{What steps are necessary to bring a \ms{} driven \og{} ``from code to
the cloud''?}

``From code to cloud'' includes all the steps necessary to build and deploy the
complete \ms{} \og{} application. In a first step the application must be built
in a cloud-affine way by respecting by respecting all \ms{} tenets. If the
application does not suffice the tenets it is still possible to refactor the
application to make it \ms{} friendly. O.Zimmermann provides a comprehensive
proposal on architectural refactoring for the cloud
\cite{zimmermann2017architectural}. Once an application satisfies all tenets it
is suitable for cloud deployment.

A requirement for cloud deployment is that the application code is fully
functional and tested locally. It is also assumed that the target environment is
correctly set up meaning the container engine is running in cluster mode and all
the required software like the database and the message broker are installed and
accessible to the cluster. From this point the necessary steps are:

\begin{itemize}
  \item Locally build the binary executable file of each \ms{}.
  \item Locally build a Docker container for each service containing the service
  executable file that was built beforehand along with all required dependencies.
  \item Test the composed application locally using local deployment launch
  configurations.
  \item Upload the complete application code to the target environment using ftp
  or clone it from Git via SSH.
  \item Build the application on the target environment using the same
  configuration that has been locally used.
  \item Start required third party software which is not integrated into the
  application.
  \item Start the application using the container engine CLI via SSH.
\end{itemize}

\noindent
\textbf{How can the deployment process be automated?}

Maven and Docker are great technologies to reduce and automate the number of
steps performed by the developer to deploy a \ms{} application. The launch
utility of MicroNet combines both techniques by automating the generation and
execution of the docker-compose and the master .pom file which further reduces
the number of manual steps.

The MicroNet \textit{Launch Utility} locally provides shortcuts for all
deployment steps. The process of uploading and starting the application on the
target environment varies greatly according to the chosen cloud provider. My
recommendation for cloud deployment is therefore to execute the necessary
console commands by hand on the server terminal via SSH. The MicroNet
documentation provides a guide on how to do this \cite{micronet2017doku} for a
MicroNet application.\\

\noindent
\textbf{How well do \ogs{} run in a cluster environment?}

As already mentioned in \autoref{sub:usability_ms_og} it was difficult to
gather real world performance data on \ms{} driven \ogs{}. But despite this fact
the few play tests that where conducted using the Spacegame prototype showed that
MicroNet works in general.

Several short play tests over the Internet (10minutes) with testers in Berlin,
Vienna, Switzerland and Liechtenstein showed that the Networking solution of
MicroNet in general is robust enough for a unreliable Network like the Internet.

One extended test of Spacegame in a \gls{lan} with 6 simultaneous players also
went very smoothly. This was however the only larger test ever conducted with
MicroNet and is therefore not representative.\\

\noindent
\textbf{How does a modern container engine assist in deploying and operating a
\ms{} driven application?}

A big part of the effort during the course of this thesis went to the evaluation
of composition solutions of containerized applications. Containers are a great
tool to completely eliminate the need of installing dependencies. During
development of an application you will often encounter the following situation:
``Aha, there is a container available for this dependency so I dont need to
install anything, very nice!''

This general availability of dependencies as containers makes it very simple to
test new technology without going through a complex installing process. But
still the dynamic nature of containers can sometimes be a burden when it comes
to overlay networks and shared volumes. These two aspects which are essential
for containerization are quite complex even for Docker which I consider a
reasonably simple container solution.

Kubernetes and Mesos are very powerful solutions but are very complex, and
setting up an application can be quite a challenge. My recommendation is to use
Kubernetes if enough resources are available to set up such a system from
scratch or if a cloud service provider like \gls{aws} or \gls{gcp} is used to
deploy the application. These providers natively support Kubernetes
applications. For small teams I recommend using docker-swarm in a first step
because it is easy to get started with. It is later possible to translate a
docker-swarm application to Kubernetes with a conceivable effort.\\

\noindent
\textbf{What tools exist that help the developer with the deployment?}

Except for development tools for Web applications I did not find many tools that
specifically assist in the deployment process of generic \ms{} applications. I
believe this is because of the very diverse nature of \ms{} applications beside
Web applications. Since each service is individually developed and they are
usually composed using RESTful HTTP no tools are needed to control the overall
deployment process of an application as long as the \gls{api} of the application
is available via a Web \ms{}.

MicroNet challenges this practice by providing a set of tools to specifically
simplify the deployment of a \ms{} application as a whole. Basically the whole
MicroNet toolset helps in this regard.

\subsection{\ms{} Composition}

\noindent
\textbf{What degree of composition is necessary for a \ms{} application
to work?}

The MicroNet concepts \textit{Shared Model} and \textit{Service API} are the
only concepts needed to logically compose a \ms{} application. The
\textit{Shared Model} is needed to provide a common data access layer
facilitating session management and a durable persistence functionality. The
\textit{Service API} concept is needed to define the format and the semantics of
the message transfers.

For physical composition it is enough to rely on a composition engine like
Docker Swarm. Since these engines are good at abstracting from physical
environment circumstances I always recommend using a composition engine. The
message broker is also an important part of the physical composition concept
because it allows services to find each other via named message queues.\\

\noindent
\textbf{Which of the common paradigms of orchestration and choreography is
better suited for \ms{} composition?}

For physical composition orchestration is very popular since most composition
engines are orchestration solution (all three evaluated). I am not aware of a
composition engine that uses choreography for physical composition. 

For logical composition, choreography is a good solution. Choreography is a very
abstract concept. One way to think of it is that a \ms{} never knows the state
of its environment by not knowing if another service is available but to assume
it. This design requires the intense use of time-outs, retries, and roll-backs
to detect and relieve failures of \mss{}. In my opinion choreography is hard to
get right but provides the best results in application robustness and scaling.

The MicroNet concepts \textit{Shared Model} and \textit{Service API} are
concepts which are inspired by choreographical composition. Also the loose
coupling achieved through the message broker facilitates choreography.\\

\noindent
\textbf{Which middle-ware is most useful in providing \ms{} composition?}

In regard to the three evaluated composition engines I consider Docker Swarm as
the most simple one to learn but with a shortage of convenience features.
Kubernetes offers a lot of built-in functionality to the price of a much more
complicated application provisioning process. Apache Mesos, although not
examined in detail, appears to be somewhere between the two. It offers many
features and is reasonably simple but I found the documentation not as
comprehensive as the other two.\\

\noindent
\textbf{How can multiple game features be developed in parallel in different
\mss{} while preserving the overall application behaviour?}

\mss{} must first of all define a clean fine grained interface through the
\gls{api}. MicroNet then provides assistance via \textit{Code Assist} in
offering the \textit{Service API} to developers of other \mss{} who then can
access the \gls{api} via MicroNet messaging. This is done to rely on
functionality provided by other \ms{} teams without knowing any implementation
details and to use the networked \gls{api} functionality like you would use a
library via local function calls.

Since the polyglot persistence tenet allows database implementation to be
excluded from the general application design, this encourages independent
development on persistence solutions. However in an \og{} the base data of the
application is accessed very frequently by basically all services and therefore
the concept of a session store offers a \ms{} friendly solution to persistence
in distributed applications.

MicroNet softens the polyglot persistence tenet by providing an application wide
session store via Couchbase. The session store allows uniform access over all
the application according to the types defined in the \textit{Shared Model}.\\

\noindent
\textbf{What problems are introduced by loosely coupling game features?}

The loose coupling of game features forces the developer to think very early
about service separation and service interfaces. A change in these architecture
critical decisions can involve multiple services.

One critical point of coupling in an \og{} application is player specific data,
since player data is the integral part of almost every game and is often closely
related to the whole game application. It is difficult to split such a central
domain into multiple sub-domains. The solution is to use a mechanism like a
shared caching mechanism like the session store to provide the player specific
data to the services. This solution provides the necessary performance and does
not break the \ms{} principles.

Another problem is that splitting a feature into multiple services inevitably
increases the amount ob bandwidth required in the application internal network.
Usually this should however not be a problem since the internal networks are
mostly \glspl{lan}.\\

\noindent
\textbf{How does the consistency vs. performance trade-off impact \ogs{}?}

Fast-paced games are very performance critical. This regard has been discussed
in detail in project thesis one \cite{biedermann2015project1} in section 3.3
Network Game Technology. In the present thesis the performance issues have been
simplified by separating the requirements for the performance critical real-time
simulation of the game from the back-end functionality of the game which is not
that performance critical. Similar to other applications the performance of
\ogs{} can be improved by profiling the application and identifying bottlenecks
which can be improved.

The consistence requirements of \ogs{} can also be separated into two groups:
strong consistency for important transaction (when they include real money
transaction, or significant time effort by the player involved) and
non-important transactions (moving the avatar in a MMO).

There is always a trade-off between consistency and performance as stated by
various sources \cite{wada2011data}, \cite{olston2000offering}, and
\cite{franklin1997transactional}. MicroNet emphasises on performance whenever
possible based on the assumption that players in general are very modest and
will tolerate minor discrepancies in the game if they have fun in general. For
financial relevant transactions however performance can be disregarded and full
attention must go to realize strong consistency.\\

\noindent
\textbf{What protocols help to reduce service coupling?}

Usually \mss{} provide a RESTful \gls{http} \gls{api} to maximize availability.
In the case of REST I agree with the common opinion that it is the way to go for
\mss{}. In regard to the way REST is realized I prefer a message broker over a
web service. This is because the message broker offers much more flexibility with
the message format and provides functionality like persistence and topics, which
a web server does not. Also it is possible to implement REST only partially
while still achieving a similar loosely coupled design. The divergence to pure
REST has been discussed in project thesis one \cite{biedermann2015project1} in
section 4.5.3 Common Communication Mechanisms.\\

\subsection{\ms{} Development}

\noindent
\textbf{What separates \ms{} development from regular development?}

As this thesis indicates, \ms{} composition is the big addition to regular
monolithic application development. The major challenge in this regard is to
develop functionality in a distributed way which is suitable for \ms{}
application (cloud affine). An explicit task in this regard is to use the API of
foreign services as their developers intended to. Using such a distributed
API always introduces asynchronism so the developer is always forced to consider
data consistency requirements for the application. Using eventual consistency
concepts like timeout, retry, and roll-back is a convenient way to tackle 
consistency requirements as explained in \autoref{sub:composition_concepts}.

Distributed application design inevitably leads to distributed errors which
span over multiple \mss{}. Finding such errors can be costly since often
multiple developers of different services are involved. The collaboration of
multiple development teams for a single application further intensifies this
challenge.

Also networking is always an integral part of every \ms{} application and it can
be cumbersome to develop a good networking infrastructure. This is further
hindered by the different environments a \ms{} application has to run on like
the development computer or a cloud environment. Application containerization
further complicates network design because it involves designing overlay
networks on top of the physical network infrastructure.

Keeping services stateless is another discipline which heavily influences the
development of a \ms{} application. Compared to monolithic application the use
of classical member variables to store the application state is not feasible for
distributed applications. As soon as multiple instances of a service are
deployed in parallel out of scaling needs variables in memory are in sole
possession of one instance only which leads to state divergence.  The use of
mechanism like the MicroNet \textit{Session Store} concept allows to work around
this limitation but introduce a significant overhead in development.

Lastly the initial deployment of a distributed application is major challenge.
All required software like networking and database solutions have to be set up
properly on the production environment. The initial deployment of the
application itself also quite troublesome since many services have to be
deployed simultaneously. However once the application is deployed successfully,
patching it is a lot simpler than patching a monolithic application.\\

\noindent
\textbf{What tools help with \ms{} development?}

The MicroNet tools in this regard are specifically designed to help with \ms{}
development. 

Additionally the Docker Tools for Eclipse are very helpful to quickly manage the
docker engine by providing wizards to build and run containers. The Docker Tools
spare the developer the need to execute every Docker interaction via the command
line which is a great simplification in my opinion. It also provides a view to
explore which containers and images are currently present on the Docker Daemon.

SourceTree is a very comprehensive Git tool and greatly helps to keep track of
\ms{} application version control since they often extend over multiple Git
repositories.

