\chapter{Conclusion}

In this concluding section of the thesis I want to give my own opinion about the
topics \mss{} and \ogs{}. My own view on the topics is heavily influenced by my
background as a programmer and hobby game developer.

\section{Summary of Findings}

The research motivation for this thesis is \ms{} development, especially
\ms{} composition and deployment. In these two areas I gathered a lot of
knowledge during this thesis and in this section I want to share the findings
that i find most relevant. These findings represent my attempt to answer the
knowledge questions in \autoref{sub:hypothesis}.

\subsection{Usability of a \ms{} for Online Games}

\noindent
\textbf{Can an Online Game run in a \ms{} environment?}

Yes, it is possible to run an \og{} in a \ms{} environment. I has already been
done in the industry (\cite{pronschinske2015turbine}) and MicroNet is an example
on how to accomplish this.\\

\noindent
\textbf{Is a Microservice influenced architecture suitable to design an Online
Game?}

The design of a \mss{} application heavily influences the way \ogs{} are
developed. This type of architecture forces the developer to split up the game
logic into small shard of \ms{} ``size''. This enforces that large domain chunks
which have high cohesion (like for example all player related functionality)
have to be split into smaller chunks. It is sometimes difficult to achieve
needed functionality this way. A good example of this aspect is the shop service
which served is explained in detail in \todo{autoref thesis 2}.

\noindent
\textbf{Is the added complexity during development tolerable?}

\ms{} indeed  adds an overhead during development. This effort pays off later
since scaling and extending of the application are simplified greatly. The added
effort occurs mainly at the beginning of a game project for example to set up
the continuous integration work-flow. For large teams (AAA companies) this
overhead can be neglected since a DevOps team can positioned to cope with these
boiler plate tasks beforehand and during the project.

For small teams (independent developers) however the overhead can be quite
troublesome. Once the development environment for \ms{}is set up the effort is
equivalent to regular \og{} development. MicroNet is aimed to fill exactly this
gab and make it easier to start with \ms{} driven \og{} development.

\noindent
\textbf{Do \mss{} have enough performance to drive a fast-paced \ogs{}?}

This question is very hard to answer in the context of this thesis. To give
sophisticated assumptions about the performance of a \ms{} application a  
full-fledged test game and a large enough group of testers is needed. The
Spacegame which I developed in my free time is not compete enough to fulfill
this role. During this semester i had basically no time at all to improve the
stability of the Spacegame to a level suited for TAR. Also the acquisition of
testers can be quite challenging since it involves a lot of community work
which is out of scope for this thesis.



\subsection{Reasons to use \mss{}}

\subsection{Tools are a Big Help}

\subsection{Achieving Platform Independence}

\section{Conclusions Drawn my Results}

\subsection{Hypotheses Verification}

\subsection{\ms{} Tenet Guidelines}

\subsection{Contradictions to a Pure \ms{} World}

\subsection{Questions for Practitioners}

\section{Recommendations for Further Research}