\section{Conclusions Drawn by Results}

This conclusion section represent my proposal for the most important points that
need to be considered when an \og{} is realized with \mss{}. It summarized all important
aspects of this thesis which are relevant fore \ms{} \og{} development.

\subsection{\ms{} Tenet Guidelines}

I suggest using the \ms{} tenets to ensure that an \og{} application is designed
according to common \ms{} principles. In this section I provide guidelines
on how to suffice each tenet.

\subsubsection{Fine-grained Interfaces and Single Responsibility}

A fine grained interface to a game application can be achieved by directly
exposing the API offered by the \mss{}. This allows users and other  \mss{} to
use the interface in a very specialized way by only accessing the functionality
the truly need. The major challenge in to make the interface available to all
services and clients.

Several MicroNet components are dedicated to facilitate fine-grained interfaces.
The MicroNet annotation tool allows to define the interface. MicroNet
code assists offers the interface to the developer. MicroNet annotation
processing exports the API and automatically integrates the API into the service
executable. The API gateway is responsible to enforce security policies to
restrict external access to parts of the API.

The single responsibility aspect must be tackled by choosing the ``right''
service granularity. Right means that the service must be neither be to small or
to large\footnote{100 lines of code appeared to be a good upper bound as an
indicator that a certain service is to large.}. The granularity is also heavily
influenced by the DDD tenet.

\subsubsection{Domain-Driven-Design (DDD)}

I consider domain driven design a technique which makes software development
significantly easier to understand. To design a \ms{} according to technical
domains like for example a database service or a serialization service
makes little sense for most applications. Instead the domain driven approach
helps greatly to understand the application as a developer. DDD is also a useful
approach to achieve service confinement and to define the service API.

With the shared model and the service catalogue, MicroNet provides an
implementation approach to DDD. Both components help the developer to define
the application according to the problem domain (the game under development).

\subsubsection{IDEAL}

\paragraph{Isolated State}

Making game services stateless is a task that is up to the developer of the
\og{}. It requires discipline to achieve a truly stateless system. Any data that
must be persisted must always be stored in a external storage solution like a
relational or NoSQL database. This is the only state representation allowed in a
truly stateless \ms{} application. Also every message transfer must be stateless
and therefore fully confined in itself.

MicroNet softens the stateless tenet by allowing synchronous messages (request
response). This allows services to wait a short amount of time for a response
that is expected to be available directly. In cases when the processing service
needs longer to process a request than the timeout it must send a notification
to the requesting peer to convey that the result will be sent later
asynchronously. The problem here is that the asynchronous result might never be
sent, maybe because the processing service fails, which can lead to game state
corruption because of lost transactions. Persistent logging of every
request to a durable storage is the only way to cope with this problem.

The session store is a concept used by MicroNet to persist session information
and make it available to multiple services. Since this approach decouples the
game data from the game logic it makes it easier to keep services stateless. The
drawback is the overhead in development to store and load data from the session
store instead of just having the sate in memory. 

Finally the fact that game sessions are always state-full in general has an
influence on state isolation of a \ms{} \og{} application. Game sessions being
state-full implies that the simulation services running the sessions are
state-full as well. Keeping track of state-full services using game session
management adds additional complexity to the system because each of these
services must be reachable individually. The game session concept of \mn{}
defines the process that is necessary to start, stop, monitor and access game
sessions running on simulation services.
	
\paragraph{Distribution}

The distribution aspect is already tackled by using a \ms{} application design.
\ms{} applications are distributed system by definition.
	
\paragraph{Elasticity}

Elasticity can be guaranteed for all services which allow to run multiple
instances simultaneously. Since these services are all stateless an infinite
number can be deployed in parallel to support varying player demands. 

But there might be cases that a service may only exist once (for example the
round service of the sample game \autoref{sub:example_game}). These singleton
services are not elastic because they cannot be deployed multiple times.
Due to this reason the singleton services must be as minimalistic as possible
and it must be able to restart them on failure without loosing the game state.

Since the persistence layer is extracted from the \mss{} it can introduce a
hidden bottleneck. To cope with this the scaling functionality for the used
storage technology must be used. MicroNet uses a Couchbase NoSQL database which
can be scaled very well maintaining a quorum for data consistency.

A relational database always represents a bottleneck. But also relational
databases can be scaled up to a useful level in cases a relational database is
necessary for an application. Having multiple independent relational databases
on multiple physical host each dedicated to only one problem domain
according to the DDD tenet (like for example a User Database) can also help to
spread the load out to multiple databases.
	
\paragraph{Automated Management}

The automation of containerized application management is very excessive task
and falls into DevOps business. Many existing tools help in this regard: Docker
and Maven both offer a CLI which is very well suited for automation with shell
scripts. Git allows to automate the source control management and also offers a
CLI that can be automated using scripts. Although not evaluated in this thesis,
Jenkins provides a way to completely automate the build and test process.
Jenkins supports Git and Maven and is therefore very well suited for MicroNet.

The MicroNet tools also provides a selection of utilities to help automate \ms{}
application management. The Launch Utility provides shortcuts compose the
application and to execute the build process and the service explorer provides a
UI to initiate any automation scripts.

\paragraph{Loose Coupling}

Loose coupling of \mss{} is the key to make them extendable and replaceable.
Loose coupling can be achieved by logically composing the services using a
shared mechanism and leaving the implementation to the service. The physical
composition which might introduce service coupling can be abstracted using a
composition engine which can be seen as a black box to minimize coupling.

MicroNet achieves loose coupling by relying on the message broker as the only
physical dependency. As soon as a service can communicate with the message
broker it is able to participate within the \ms{} application. 

In addition to receiving messages, services must also understand messages.
In MicroNet this is achieved by using the shared model to make message semantics
known throughout the application.

The session store is another concept of MicroNet with encourages loose coupling.
It allows services to store data and not to care about how it is stored.
The format of the data in the session store is based on the shared model.
This allows services to see the session store as a black box and to just query
the typed data they care about.

\subsubsection{Polyglot Programming and Persistence}

The polyglot programming and persistence tenet is very helpful to introduce
foreign technology in an application. It allows to choose the best suited
technology for every problem while maintaining the overall application
behaviour.

MicroNet offers two persistence solutions to showcase polyglot persistence.
PostgreSQL as a durable relational database to allow strong consistency and
Couchbase as a NoSQL database for eventual consistency and session management.

The polyglot programming strategy is used by MicroNet to also support C\#
besides Java. C\# is the programming language used to develop Unity3D games. The
integration of a service written in a foreign programming language requires that
the MicroNet Networking functionality is available in the language.
ActiveMQ which is used for MicroNet messaging offers many access possibilities
which allows the integration of many programming languages within MicroNet
applications.

\subsubsection{Lightweight Containers}

Containers are the enabling technology for \ms{} applications. Without
containers it would be nearly impossible to maintain the state of a \ms{}
application.

Docker is the number one container engine in this regard and can be considered a
defacto standard. I have to admit that I did not evaluate other container
technologies due to the dominance of Docker. I believe Docker will maintain
strong position because it is improved permanently.

The benefit I appreciate the most about containers is the ability to download
and start dependencies quickly without going though a complex installation
process. This makes it very easy to evaluate new technologies.

\subsubsection{Decentralized Continuous delivery}

I did not examine this topic as intensive as other topics. The reason for this
is that MicroNet was never meant to reach production level where continuous
delivery would be mandatory.

Although in my opinion a \ms{} application designed according to the \ms{}
tenets is very well suited for continuous delivery. Services can be replaces
very easily even during run-time by performing a rolling update. New
functionality can be added dynamically by introducing new services.

One \og{} specific aspect is the delivery of the game client which containing
all graphical assets to display the game on the client. When service
functionality changes it can be necessary that a new client version needs to be
shipped. The patching is very noticeable to the players since the game client
can become quite large for rich \ogs{}.

\subsubsection{DevOps}

During the development of a \ms{} application DevOps tasks are very frequent.
The responsibility to bring to the production environment and to maintain the
application is a very excessive process also after application release.

Because each service can have multiple dependencies that each must be compatible
with the application the DevOps effort grows with the application. 

DevOps work is mostly command line work which I could experience first hand
while evaluating composition and deployment technologies for \mss{}. In the end
it all boils down to excessively using the terminal to configure the
application. 

The DevOps work can be reduced by developing tools to automate the DevOps tasks.
MicroNet does this by generating scripts automatically. Therefore the DevOps
engineer only must execute the scripts instead of additionally writing them by
hand.


\subsection{Questions for Practitioners}

In his positioning paper\todo{cite olaf} O.Zimmermann provides nine questions
for practitioners to gather experiences about \ms{} concepts and to provide a
common discussion basis. Although these questions are mainly aimed at developers
in the software industry my findings during this thesis can also contribute a
proposal to the practitioner questions.\\

\textit{1. Can you share any experiences and/or hint how to ``sell'' an
investment in microservices to business stakeholders(e.g., project sponsors, product managers,
C-level man-agement in business and IT organizations)? How well do these
experiences and hints relate to and/or align with the microservices tenets?}\\

The benefits which of \mss{} offer in regards to maintenance and scalability
will not immediately pay off at project release but during the lifetime of the
application. But it is hard to explain to business stakeholder because the might
not know the effort needed to maintain legacy applications. One very positive
experience is that working services are very stable and can stay unchanged over
a long time due to service isolation. This allows developers to only work on
small parts of the application while the majority stays unchanged. 

\mss{} are also very well suited to organize large development teams for large
projects into smaller independent teams. This allows for much more control over
the size of development teams.\\

\textit{2. In which business domain and socio-technical context(or application
genre and software operating range w.r.t. quality attributes) have you applied
microservicesconcepts and technologies - either successfully or
unsuccessfully?}\\

This thesis is an example of how to apply \mss{} to the game domain. It appears
that the combination of the two works quite promising.\\

\textit{3. Which microservices principles (e.g., ``independently deployable
services'') did you use in your microservices architecture designs, and how did you
implement them (patterns, frameworks, middleware, tools)? Did you deploy to
public or private cloud offerings (e.g. AmazonLambdas, Google Cloud Function) or to
more traditional application hosting environments?}\\

The \ms{} which this thesis is based upon are the \ms{} tenets defined by
O.Zimmermann. The goal of this thesis was to respect all \ms{} tenets.
MicroNet serves as a reference on how to implement all tenets.

The deployment process is also cover by MicroNet especially the MicroNet tools.
Both Amazon Cloud and Google cloud have only evaluated via literature research
and therefore no statement about the quality of these providers can be made.

\textit{4. How do you see the relationship between REST and microservices? Is
the usage of Web protocols required and sufficient? Or is RESTful HTTP only one of several
valid remote communication options in the microservices architect�s toolbox and,
if so, what are the decision drivers when choosing an option?}\\

REST is in my opinion a requirement to achieve the desired stateless \ms{}
application behaviour. But for the game domain the traditional RESTful HTTP
approach is just not suited. A more flexible and optimization friendly
communication solution is needed. A message broker fulfills this role perfectly.
The main drivers behind this decision is the persistence which a message broker
offers along with functionality like topics and connection handling. 

\textit{5. How did you find an adequate/a suited service cut (e.g.,how
small/fine is small/fine enough)? How can DDD (and/or other approaches to application
scoping and functional partitioning) be applied to decompose monoliths into
services�practitioners have reported that they would welcome guidance that is
more concrete than the rather frequently stated advice ``define a bounded
context for each domain concept to be exposed as service''?}\\

Service size is the factor that matters the most in my opinion. A small service
is easy to maintain and a new developer can gasp the functionality of a \ms{}
quickly. With small I mean that the service size should be less then 2-3 IDE
screens.

\textit{6. How did you overcome ``distribution classics'' design challenges such
as service lifecycle management, data representation/schema mismatches, service
versioning and evolution (e.g., change of interface in terms of syntax and/or
semantics), and error handling on your projects?}\\

The distribution classics are mainly covered by the MircoNet composition
concepts. The Shared Model and API represents the solution to overcome
representation/schema problems. Service versioning and evolition is supported by
the service explorer and lifecycle management is taken care of by the
composition engine.

\textit{-- Did you define machine-readable service contracts? If so, what
should they cover and how should they be expressed (e.g., are all REST maturity
required,including support for Hypertext as the Engine of Application State
(HATEOAS), is billing information included)? If not, how did you achieve
syntactic and semantic interoperability between service consumers and
providers?}\\

HTTP is not used at all within MicroNet at the moment. Instead the service
contracts are defined by the service API.  

\noindent
\textit{-- How did you deal with audit requirements, e.g.,Completeness,
Accuracy, and Validity of, as well as Restricted access to financially relevant business
objects (e.g., CAVR controls)?}\\

MicroNet guarantees the consistency of financial business object by relying on
strong consistency of a relational database using ACID transaction. But since
this kind of transaction is only a minor part for \og{} this was not covered in
detail in this thesis.

\noindent
\textit{-- Should overall, end-to-end data integrity be ensured in microservices
architectures, either centrally or decentrally? If so, how to manage views,
foreign key relationships and other semantic links across microservice
boundaries? And how to backup an entire service landscape at once (atomic system
snapshot, incremental backup)?}\\

I suggest to completely rely on eventual consistency whenever possible. Eventual
consistency is offered by NoSQL databases like Couchbase for example. End-to-end
data integrity is only needed for financially relevant actors as mentioned
above. Since Couchbase is document oriented it is schemaless foreign keys can be
modeled acoording to the needs of the developer. Micronet shares the semantics
over service boundaries using the shared Model.

\textit{7. Do you have any advice/guidance how to compose microservices into end
user client applications? How about application-level intermediaries, i.e., can
microservices also be clients of other microservices? If so, how to avoid
microservice deployment dependency and dynamic invocation �spaghetti� (e.g.,
cycles, overly deep invocation chains)? Do you see a need for/can you recommend
any tools, libraries, frameworks, middleware that can assist with this task, or
is plain old development(applying state-of-the-art software engineering
practices) sufficient?}\\

\ms{} composition is one major topic of this thesis but it is difficult to give
a all mighy answer to this question. Microservices can indeed talk to each other
and I find this approach very useful to distribute application behaviour over
multiple services. I can recommend Docker, Docker-compose and docker swarm as
the most helpful tools for \ms{} composition. 

\textit{8. Can you report on your technical and organizational scaling
strategies (e.g., when having to deal with large services landscapes and rich/complex domain
models with hundreds or thousands of interconnected entities)? Which tactics and
patterns support these strategies well?}\\

This question cannot be anserwed by a thesis if this small scale.

\textit{9. Which research and development challenges for a broad and sustainable
adoption of microservices can the service-oriented computing community derive
from your experience?}\\

To cope with the statelessness of a \ms{} application is the biggest challenge
during development. This goes hand in hand with the consitency problems that
will arise inevitably.

Another challenge is to deal with low level networking for the application. A
network wise complex application like an \og{} need a vast number of ports and
protocols that must be suported and configured. The container engine as an
additional layer futher complicates this problem.

Also the address shortage of IPv4 address makes it difficult to develop
networked applications for the future. Almost all networking solutions rely on
IPv4 technology but you cant just go to the supermarket and buy one. This
problem might intensify over time. 
