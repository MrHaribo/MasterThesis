\section{Recommendations for Further Research}

Several topics could not be treated during this semester mostly because of the
time constraints.

\subsection{Shared Model Logic Code Injection}
\label{sub:model_code_injection}

One limitation of the shared model is that the generated model classes (POJOs)
do not allow adding any logic to them. This would however be helpful in order to
share functionality in a way that is decoupled from any programming language.
One possibility to inject code into model objects is to use a high level
scripting language like Java Script or Lua. Interpreters of scripting languages
are available on many different platforms and there is always the option do
define a proprietary scripting language optimized for the problem domain.
Interpreted scripting languages are a effortless and portable way to make
application logic available throughout a \ms{} application.

Although it has to be mentioned that interpreting Java Script code always
presents a security issue. This aspect has to be examined with absolute care.

Code Injection using a high level scripting language could also be a convenient
way to introduce \gls{pcg} techniques fluently into the \og{} development
process.

\subsection{Shared Model Automatic Database Integration}

One purpose that the shared model is designed for is to make it automatically
compatible with the database solution. This is not implemented in the current
version of MicroNet. The proposed design for this feature is to provide a
DataAccess layer via the MicroNet framework which offers a comprehensive API to
query and update objects in the database using the shared model as a the driver.

\subsection{Message Listener AST parsing}

Momentarily MicroNet annotations are quite cumbersome because to fully describe
a message listener all the used message parameters have to be listed.
This aspect could be simplified by parsing the \gls{ast} of the message listener
and to perform reference counting of parameters to automatically deduce
parameters of a message listener.

\subsection{Game Engine Game Model Integration}

The shared model is additionally designed in a way that is compatible with the
game graph data structure of game engines. It would be possible to realize an
automated integration of the MicroNet shared model within the game engine game
graph. In a system of this nature game objects which are currently part of the
game simulation could be automatically updated using an event based messages.
The game simulation then could make use of so called views of game entities and
their properties which are automatically updated by the framework. This way the
developer is spared with manual synchronization of the game state and only needs
to contribute code in a framework like fashion to cover events happening for
views. An example uf such an event could be a \textit{onHealth} event which
would cause the game client updates the health bar visuals accordingly. This
dynamic entity view feature would greatly simplify the networking implementation
of the game simulation.
