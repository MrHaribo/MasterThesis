\section{Recommendations for Further Research}

Several topics could not be treated during this semester mostly because of the
time constrains.

\subsection{Shared Model Logic Code Injection}
\label{sub:model_code_injection}

One limitation of the shared model is that the generated model classes (POJOs)
do not allow to to add any logic to them. This would be helpful to share
functionality in a way that is decoupled from any programming language. One
possibility would be to use Java Script as an interpretable language which can
be injected into model classes. Java script interpreters are available in many
languages which would allow a very portable way to make application logic
available in a \ms{} application.

\subsection{Shared Model Automatic Database Integration}

One purpose that the shared model is designed to is to make it automatically
compatible with the database solution. This is not implemented in the current
version of MicroNet. The proposed design for this feature is to provide a
DataAccess layer via the MicroNet framework which offers a comprehensive API to
query and update objects in the database using the shared model as a the driver.

\subsection{Message Listener AST parsing}

Momentarily the MicroNet annotations are quite cumbersome because to fully
describe a message listener all used message parameters have to be listed. This
aspect could be simplified by parsing the \gls{ast} of the message listener and
perform reference counting of parameters to automatically deduce parameters of a
message listener. 

\subsection{Game Engine Game Model Integration}

The shared model is also designed in a way that is compatible with the game
graph data structure of game engines. It would be possible to realize an
automated integration of the MicroNet shared model into the game engine game
graph. In a system of this nature game objects which are currently part of the
game simulation could be updated automatically using an event based messages.
This feature would simplify the networking implementation of the game simulation
greatly.
