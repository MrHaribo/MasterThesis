\section{MicroNet}

\subsection{Description of the MicroNet Framework}

Microservices reduce the complexity of an online game by splitting the game
domain into small shards that can be developed independently. These shards are
established by a domain driven design (DDD) approach. Each type of Microservice
is repsonsible to handle all aspects of the domain shard completly (cohesion).
This reduces the need to invest effort in inter service.\\

How does MicroNet help the developer to develop large online games:
\begin{itemize}
  \item MicroNet does not reinvent the wheel for online game development
  \item MicroNet works in conjunction with existing technologies, specifically popular game engines
  \item MicroNet tries to make the developer aware of the problems that are encountered during the development process
  \item While using MicroNet the developer learns the process of online game development 
  \item It helps the developers to get the difficult tasks like consistency of a
  asynchronous system right
\end{itemize}

What does MicroNet provide in this regard:
\begin{itemize}
  \item Provides a clear architecture for the game back-end
  \item Allows to develop individual game features independently
  \item Provides a simple protocol definition that helps the developers to loosely couple game features
  \item Offers standard solutions for technical requirements: Messaging
  functionality and Persistence functionality	
\end{itemize}

\subsection{Eclipse Tools}

Since MicroNet makes usage of various thirs party libraries and tools there
needs to be a way to manage them all. This purpose solves the Eclipse tools for
MicroNet. The MicroNet Tools is a set of plug-ins that are partially independent
of each other. They all help the developer to implement game specific \mss{}.

Since the UI of an Eclipse plug-in is written with the SWT library it can easily
be exported to a standalone tool that does not require Eclipse. However any code
related feature do require the Eclipse IDE.

\subsubsection{Service Generation}

Allows the developer to not worry about boiler plate code of setting up the
service in the framework. This covers the setup of the appropriate networking
solution according to the environment. This allows the developer to focus on the
actual service code.

\subsubsection{Code Assist}

The Code Assist plug-in helps the developer to keep track of functionality
provided by other services. It allows a type-safe communication between
services. The information that is needed to give these API proposals is
extracted from the service implementation at compile time. The information is
then shared via the version control system. This process is manually and the
developer must check in and probably merge his changes. 

The format of the distribute API description is not relevant because it is a
machine-readable format and the developer sees it in a polished form.

\subsubsection{Management UI}

The management UI plug-in helps the developer to get an overview over the whole
game application. It helps the manage the versions of services of the game
services as well as the versions of dependencies. This plug-in highly relies on
the description of the maven projects.

\subsection{PCG Solutions for Online Game}

\subsection{A Solution for \ms{} Development}