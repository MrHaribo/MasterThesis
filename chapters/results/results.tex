\chapter{Results}

\section{Solutions to the Research Problems}
\subsection{Order of Presentation}

\section{Descriptive Analysis}

\subsection{\ms{} Tenet Specific Solutions}

\subsection{Contradictions of the Solutions in a Pure \ms{} World}

\subsection{Description of the MicroNet Framework}

Microservices reduce the complexity of an online game by splitting the game
domain into small shards that can be developed independently. These shards are
established by a domain driven design (DDD) approach. Each type of Microservice
is repsonsible to handle all aspects of the domain shard completly (cohesion).
This reduces the need to invest effort in inter service.\\

How does MicroNet help the developer to develop large online games:
\begin{itemize}
  \item MicroNet does not reinvent the wheel for online game development
  \item MicroNet works in conjunction with existing technologies, specifically popular game engines
  \item MicroNet tries to make the developer aware of the problems that are encountered during the development process
  \item While using MicroNet the developer learns the process of online game development 
  \item It helps the developers to get the difficult tasks like consistency of a
  asynchronous system right
\end{itemize}

What does MicroNet provide in this regard:
\begin{itemize}
  \item Provides a clear architecture for the game back-end
  \item Allows to develop individual game features independently
  \item Provides a simple protocol definition that helps the developers to loosely couple game features
  \item Offers standard solutions for technical requirements: Messaging
  functionality and Persistence functionality	
\end{itemize}

\subsection{PCG Solutions for Online Game}

\subsection{A Solution for \ms{} Development}

\section{Results of Statistical Testing}

\section{Interpretation of Statistical Results}