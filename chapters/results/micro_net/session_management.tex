\subsection{Session Management}
\label{sub:session_management}

MicroNet provides session management as a high level feature to tackle the
integral part of state-full sessions in an \og{}. Player sessions are
essential to track actions of individual players and individually persist the
progress of each player. Game sessions tackle the issue that \og{} sessions are
state-full and therefore the simulation of each game session is unique.

Session management has been one major issue throughout all three theses. The
final solution for session management relies on Couchbase as a NoSQL database.

\subsubsection{Player Sessions}

The authentication process that has been briefly mentioned in
\autoref{sub:networking} is responsible for establishing player sessions. Once
authenticated the player sessions can be used to collate incoming messages to
users. Player sessions can also be used to store frequently changing player data
to allow fast and global access. In MicroNet the player sessions can be accessed
through Couchbase documents provided by the persistence component of MicroNet.

One drawback of the solution for player sessions is the fact that with multiple
API gateways each gateway needs the ability to look up player connections to
determine if the corresponding user is authenticated. This process has to be
done upon every player message which leads to a lot of reading access to
Couchbase. A possibility to reduce the amount of database reads would be to use
a token that is created upon authentication which the user would have to send
with every request. The drawback of this method is that it introduces a
significant bandwidth overhead. It also requires the token to be shared with the
user through a secure channel which has to be established. So the current
solution is the most simple one but with the cost of many database reads. In
production the transmission of the user credentials must take place over a
secure channel. For simplicity this aspect is omitted in the MicroNet prototype
and also because this topic is well documented \cite{sun2015security},
\cite{shieh2015methods}, and \cite{thanh2016embedding}.

\begin{figure}
	\centering
  	\includegraphics[width=\textwidth]{images/architecture/PlayerSessions}
	\caption{Player session authentication process.}
	\label{fig:player_sessions}
\end{figure}

\autoref{fig:player_sessions} shows the whole authentication process with the
result of an established authenticated player connection. The message flow
consists of the following steps:

\begin{enumerate}
  \item The user sends a login request to the public broker.
  \item One API Gateway polls the message in a competing consumer fashion.
  \item The gateway forwards the login message to the internal broker using the
  \textit{mn://account/login} queue.
  \item One Account Service polls the message in a competing consumer fashion.
  \item The Account Service authenticates the user using the Account Database.
  \item The account service returns the login response to a temporary queue held
  open by the responsible gateway.
  \item The gateway service consumes the login response from the temporary
  queue.
  \item Upon login success the gateway adds a new player session to the session
  store, identified by the player connection.
  \item The gateway returns the login response to a temporary queue held open by
  the requesting player.
  \item The player consumes the login response from a temporary queue. Upon
  success he gains access to the game services (sequence: A, B, C, D\footnote{In
  step D the API Gateway must verify if the player connection is already
  authenticated. If the player is not authenticated no messages are forwarded.},
  E).
\end{enumerate}



\subsubsection{Game Sessions}

Game sessions are a strategy to deal with the state-fullness of \ogs{}. Game
sessions are unique and a player can only be in one game session at a time. Game
sessions are either infinite or terminate at specific conditions. Either way
the player must be able to find the correct simulation instance which hosts the
game session he wants to participate in. MicroNet provides this functionality
through the WorldService which is part of the service catalogue.

\begin{figure}
	\centering
  	\includegraphics[width=\textwidth]{images/architecture/GameSessions}
	\caption{The join process of a game session.}
	\label{fig:game_sessions}
\end{figure}

\autoref{fig:game_sessions} shows the process how a game session is joined.
The join process consists of the following steps:

\begin{enumerate}
  \item The player sends a join request for game session B to the API
  gateway via the public message broker.
  \item The gateway service forwards the request to the \textit{mn://world/join}
  queue via the internal message broker.
  \item One world service polls the message in a competing consumer fashion.
  \item The world service checks if the session already exists in the session
  store. If not, the world service adds the game session to the session store,
  marks it as \textit{isOpening} and initiates the game session opening process (capital
  letter sequence in \autoref{fig:game_sessions}). In this case or if the
  session is already flagged as \textit{isOpening} the world service adds the
  player to a waiting queue for the session.
  \item If the game session is open the world service directly
  forwards the IP/Port combination to the player.  if it is not open or
  just opening a ``please wait'' response is forwarded to the player.
  \item The join response is consumed by a temporary queue held open by
  the API Gateway. The gateway forwards the join response to the player
  via the public broker.
  \item The player consumes the join response using a temporary queue.
  In the case the session was already open, the player can join right away. The regular
  session flow (a, b, c, \ldots) starts now.
\end{enumerate}

In case the requested game session is not open when the player wants to
join\footnote{It is a common case that the game session is not open, for
example when the \og{} has many (thousands) of available game sessions.} the
session must be started asynchronously and the player must be notified when the
session is ready. This process consists of the following steps:



\begin{enumerate}[label=\Alph*.]
    \item The World Service sends an open region request to the
    \textit{mn://instance-open queue}.
    \item One free simulation instance polls the open region request in a
    competing consumer fashion.
    \item Once the simulation service is up and running it sends a simulation
    instance ready message to the \textit{mn://world/instance/ready} queue via
    the internal message broker.
    \item One World Service polls the instance ready message in a competing
    consumer fashion.
    \item The World Service marks the game session as \textit{isOpen} in the
    session store.
    \item The World service sends a broadcast event request to the API Gateway
    using the \textit{mn://gateway/broadcast/event} queue. The event contains
    the IP/Port combination of the simulation service which is hosting the
    game session. A list of all player IDs  which are registered for the
    session is sent along with the broadcast event request using parameters to
    make the gateway aware of all recipients of the event.
    \item One API Gateway polls the broadcast requests in a competing consumer
    fashion 
    \item The Gateway broadcasts the join event to all waiting players
    according to the recipients list. From then on the regular game session
    message flow starts (sequence: a, b, c, \ldots).
\end{enumerate}




