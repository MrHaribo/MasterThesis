\subsection{Networking}
\label{sub:networking}
Networking in MicroNet has been discussed in detail in project thesis one. This
section only provides a short summary of the topic.\\

Networking is the most fundamental aspect of an \og{} development and crucial
for the stability of \ms{} applications. The networking concept for \ms{}
applications is heavily influenced by the IDEAL tenet because it makes it
necessary to provide an abstract messaging concept that is decoupled from any
implementation.

In MicroNet this network abstraction is realized by defining a set of networking
capabilities that must be supported by the underlying network technology. These
capabilities are represented in the form of a Java interface. Standard services
can therefore communicate over the messaging interface injected by the
framework. This makes the underlying technology exchangeable. This allows
platform independence as long as the services are written in Java.

According to the polyglot programming tenet it is a requirement that arbitrary
technologies can participate in a \ms{} application. In this case an adapter
must be used to couple the external service with the application. The adapter
can be an be a implementation of the messaging interface in another language. In
this case the underlying messaging technology must provide an API for the
respective language. In the case of MicroNet ActiveMQ is used as the
message broker which provides a JMS API. JMS ports are also available for other
languages like for example the NMS API for C\#. This API is used to make the
Unity3D engine accessible from MicroNet.

To abstract underlying technical details it is assumed that all message
transfers of MicroNet are reliable (TCP). This is for the purpose of simplicity
over performance. Faster message transfers can be seen as an optimization but
this is not the topic of this thesis.

\subsubsection{Messaging System}

Messaging in MicroNet evolves around patterns: API Gateway, Reverse Proxy and
Pipes and Filters.\\

MicroNet uses a reverse proxy to interchange messages from the public Internet
with the internal network used by MicroNet. For this purpose two separate
message brokers are used and only the public broker is exposed to the Internet.
The gateway intercepts all messages from the public network and forwards them to
the responsible \ms{}. Messages are filtered by the

The reverse proxy is in fact also the API gateway. An API gateway is a standard
approach to offer the service API to the users. Upon reception of a message the
gateway can filter it according to a white- or blacklisting approach.

\paragraph{Message Parameters}

One useful feature that MicroNet provides are typed message parameters. The
parameters are defined using Parameter Codes and the Template Types both defined
int the Shared Model.

Message parameters spare the developer with the need to define a different
payload for each message transfer. For example slightly different messages can
be distinguished by parameter codes.

\subsubsection{Connection Authentication}

One challenge is to authenticate user connections. Since multiple gateways are
active simultaneously the user sessions are stored in the session store in
Couchbase. For unauthenticated connections the gateway only forwards login
messages to the login service. upon success the processing gateway adds the user
connection to the session store. Other gateways are then able to look up if a
connection is authenticated. The advantage of this solution is that it scales
very well. The downside is that it involves many reads to the session store.

A requirement for this approach to work is that the underlying networking
provides a way to correlate incoming messages with a connection. This can be
done generating a connection id hash based upon the user's IP/port combination.
ActiveMQ has this capability already built in by providing a globally unique
connection ID per connection.

\subsubsection{Messaging API}

This messaging API of MicroNet offers the following functionality to promote
stateless message transfers:

\begin{itemize}
  \item Sending a requests to a \ms{} either from a user or another \ms{}
  \item Sending a requests to a \ms{} and waiting for a response with a
  short\footnote{15 seconds seem appropriate for most short requests.} timeout
  either from a user or another \ms{}
  \item Sending an event to a user from a \ms{}
  \item Sending an event to a group of users from a \ms{}
  \item Sending a message to a topic, meaning to all subscribed \mss{}
  \item Named message parameters
\end{itemize}

These messaging capabilities are enough to model any arbitrary message transfer.
It leaves a great degree of freedom and makes message transfer implementation
very comfortable. 