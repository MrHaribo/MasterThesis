\subsection{Realizing \ogs{} with MicroNet}

\begin{figure}
	\centering
	\includegraphics[width=\textwidth]{images/architecture/GameWithMicroNet}
	\caption{The MicroNet setup for \og{} development.}
	\label{fig:game_with_micronet}
\end{figure}

\autoref{fig:game_with_micronet} shows how the MicroNet development setup for an
\og{} looks like. The game model is represented by \gls{json} files and is used
to generate the classes needed to participate in a MicroNet application. This
allows the integration of the back-end game services but also of the game
simulation server and the client into the application. Back-end can also bypass
the generated integration layer and directly use the framework functionality.

The game engine components are developed using the game engine's regular
development workflow. The integration of MicroNet can be achieved by either
mirroring the framework integration generation in the language of the game
engine or manually by programming the integration classes.

A disadvantage of this setup is that the framework integration layer has to be
realized for all the technologies used. This introduces quite an overhead, but
it is compensated by the improved work-flow which is possible once the
integration is done. It has to be mentioned that the implementation effort to
realize the code generation part which generates the framework integration
classes is rather small. Model definition and editing take care of most work in
this regard. The code generation layer is only responsible to generate simple
classes out of the well defined \gls{json} model files, which is a very
feasible task being well documented through the MicroNet code generation plug-in
which serves as a reference implementation. Also the developer always has the
choice to integrate a foreign technology into MicroNet by using only ActiveMQ as
the message broker bypassing the integration layer. ActiveMQ is easily portable
since it offers many clients on all major platforms. 

With the integration layer out of the way the developer is free to use the
MicroNet tools to speed up and clarify the development process. This includes
the possibility to integrate reference services from the service catalogue into
the application. It also includes the possibility to use MicroNet core
features like networking, persistence, and serialization.
