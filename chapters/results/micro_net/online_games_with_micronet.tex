\subsection{Realizing \ogs{} with MicroNet}

\begin{figure}
	\centering
	\includegraphics[width=\textwidth]{images/architecture/GameWithMicroNet}
	\caption{The MicroNet setup for \og{} development.}
	\label{fig:game_with_micronet}
\end{figure}

\autoref{fig:game_with_micronet} shows how the MicroNet development setup for an
\og{} looks like. The game model is represented by Json files and is used to generate
the classes needed to participate in a MicroNet application. This allows the
integration of the back-end game services but also of the game simulation
server and the client into the application. Back-end can also bypass the
generated integration layer and directly use the framework functionality. 

The game engine components are developed using the game engines regular
development workflow. The integration of MicroNet can be achieved by either
mirroring the framework integration generation in the language of the game
engine or manually program the integration classes.

A disadvantage of this setup is that the framework integration layer has to be
realized for all used technologies. This introduces quite the overhead but is
compensated by the improved work-flow which is possible after integration. It
has to be mentioned that the implementation effort to realize the code
generation part which generates the framework integration classes is rather
small. The model definition and editing take care of most work in this regard.
The code generation layer is only responsible to generate simple classes out of
a well defined model which is very doable and well documented through the
MicroNet code generation code which serves as a reference implementation. Also
the developer has always the choice to integrate a foreign technology into
MicroNet by using ActiveMQ as the message broker only bypassing the integration
layer. ActiveMQ is very portable since it offers many clients on many platforms. 

With the integration layer out of the way the developer is free to use the
MicroNet tools to speed up and clarify the development process.