\section{Validation of Survey}
\label{sec:validation_survey}

The expert opinion survey was conducted with a Google Forms online
questionnaire. Experts first had to complete the MicroNet tutorial and
afterwards had to answer evaluation questions about the following topics:

\begin{itemize}
  \item The relation between MicroNet and Java
  \item The distributed application design of MicroNet and specifically the
  \textit{Shared Model} and \textit{Shared API} concepts
  \item The adherence of the \ms{} tenets by MicroNet
  \item MicroNet, MicroNet Tools, MicroNet Tutorial, and MicroNet Documentation
\end{itemize}

First I conducted the survey with two participants on site in order to verify
that the tutorial could be completed as expected. After some minor adjustments the
tutorial and the survey were published online to reach more people. But beside
the two live sessions hardly anyone could be motivated to complete the tutorial.
At the end, the survey was completed by only three participants - a rather
disappointing fact.

Nonetheless, these three participants examined MicroNet very intensively so
their opinions can be considered as meaningful feedback.

\subsection{Results and Interpretation of the Survey}

The detailed statistical results of the survey can be found in
\autoref{survey_results}. This section summarizes the results and gives a
proposal on how these statistics can be interpreted. Each topic of the
questionnaire is discussed separately.

\subsubsection{MicroNet and Java}

All participants rated Java as a good solution for distributed application
development, and also modern language features like Lambdas and code generation
are majorly accepted. Maven is seen as moderately cumbersome, but its uses
outweigh the organizational overhead.

These results support the strong Java relation of MicroNet. Also the usage of
Maven as a main driver for build automation is accredited.

\subsubsection{Distributed Application Design}

MicroNet did not noticeably help the participants to get a better understanding
of distributed application design. This indicates the lack of a dedicated
tutorial in this regard. The \textit{Shared Model}, the \textit{Shared API}, and
the \textit{Launch Utilities} are accredited by the participants which justifies
their central architectural role.

The manual documentation of the \textit{Shared API} with Java annotations,
however, is seen as an implementation overhead since the information could also
be generated automatically by parsing the service's source code. A consequence
of manual API documentation is a divergence from the actual implementation which
inevitably leads to errors. This opinion corresponds with my own impression
which has been discussed in \autoref{sec:future_research}.

\subsubsection{\msuc{} Tenets}

The participants agree that all tenets except the IDEAL and the polyglot
programming and persistence tenets are respected sufficiently. They consider the
use of a relational database, even if it is containerized, as a violation of the
IDEAL tenet. I partially agree on this point since realizing fail-over for
relational databases can be challenging as I experienced first hand. The
participants also criticized that the polyglot programming and persistence tenet
was not implemented flexible enough.

\subsubsection{MicroNet Related Topics}

The participants were satisfied with MicroNet in general. However all
participants experienced problems during the setup process. The goal of the
MicroNet setup was to keep it simple by only relying on two software
installations: Docker and Eclipse. But the configuration of both programs
had more pitfalls than expected. Participants also liked the simple \gls{ui} of
MicroNet and the many options that the \textit{Launch Utilities} offer. Most
participants rated the performance of MicroNet sufficient. The shared opinion of
all participants is that MicroNet still needs improvements in all areas. Since
MicroNet is in a very early stage, this impression is understandable.


